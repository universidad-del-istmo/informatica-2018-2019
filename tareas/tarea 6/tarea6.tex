%%%%%%%%%%%%%%%%%%%%%%%%%%%%%%%%%%%%%%%%%
% Programming/Coding Assignment
% LaTeX Template
%
% This template has been downloaded from:
% http://www.latextemplates.com
%
% Original author:
% Ted Pavlic (http://www.tedpavlic.com)
%
% Note:
% The \lipsum[#] commands throughout this template generate dummy text
% to fill the template out. These commands should all be removed when 
% writing assignment content.
%
% This template uses a Perl script as an example snippet of code, most other
% languages are also usable. Configure them in the "CODE INCLUSION 
% CONFIGURATION" section.
%
%%%%%%%%%%%%%%%%%%%%%%%%%%%%%%%%%%%%%%%%%

%----------------------------------------------------------------------------------------
%	PACKAGES AND OTHER DOCUMENT CONFIGURATIONS
%----------------------------------------------------------------------------------------

\documentclass{article}

\usepackage{fancyhdr} % Required for custom headers
\usepackage{lastpage} % Required to determine the last page for the footer
\usepackage{extramarks} % Required for headers and footers
\usepackage[usenames,dvipsnames]{color} % Required for custom colors
\usepackage{graphicx} % Required to insert images
\usepackage{listings} % Required for insertion of code
\usepackage{courier} % Required for the courier font
\usepackage{multirow}
\usepackage{hyperref}
\usepackage{amsmath}
\usepackage{amssymb}

% Margins
\topmargin=-0.45in
\evensidemargin=0in
\oddsidemargin=0in
\textwidth=6.5in
\textheight=9.0in
\headsep=0.25in

\linespread{1.1} % Line spacing

%----------------------------------------------------------------------------------------
%	CODE INCLUSION CONFIGURATION
%----------------------------------------------------------------------------------------

\definecolor{MyDarkGreen}{rgb}{0.0,0.4,0.0} % This is the color used for comments
\lstloadlanguages{c} % Load Perl syntax for listings, for a list of other languages supported see: ftp://ftp.tex.ac.uk/tex-archive/macros/latex/contrib/listings/listings.pdf
\lstset{language=[sharp]c, % Use Perl in this example
        frame=single, % Single frame around code
        basicstyle=\small\ttfamily, % Use small true type font
        keywordstyle=[1]\color{Blue}\bf, % Perl functions bold and blue
        keywordstyle=[2]\color{Purple}, % Perl function arguments purple
        keywordstyle=[3]\color{Blue}\underbar, % Custom functions underlined and blue
        identifierstyle=, % Nothing special about identifiers                                         
        commentstyle=\usefont{T1}{pcr}{m}{sl}\color{MyDarkGreen}\small, % Comments small dark green courier font
        stringstyle=\color{Purple}, % Strings are purple
        showstringspaces=false, % Don't put marks in string spaces
        tabsize=5, % 5 spaces per tab
        %
        % Put standard Perl functions not included in the default language here
        morekeywords={rand},
        %
        % Put Perl function parameters here
        morekeywords=[2]{on, off, interp},
        %
        % Put user defined functions here
        morekeywords=[3]{test},
       	%
        morecomment=[l][\color{Blue}]{...}, % Line continuation (...) like blue comment
        numbers=left, % Line numbers on left
        firstnumber=1, % Line numbers start with line 1
        numberstyle=\tiny\color{Blue}, % Line numbers are blue and small
        stepnumber=5 % Line numbers go in steps of 5
}

\newcommand{\horrule}[1]{\rule{\linewidth}{#1}}

% Creates a new command to include a perl script, the first parameter is the filename of the script (without .pl), the second parameter is the caption
\newcommand{\perlscript}[2]{
\begin{itemize}
\item[]\lstinputlisting[caption=#2,label=#1]{#1.cs}
\end{itemize}
}

\begin{document}

\begin{tabular}{l l}
\multirow{5}{*}{\includegraphics[width=2cm]{../../recursos/logo.png}}
 & Universidad del Istmo de Guatemala \\
 & Facultad de Ingenieria \\
 & Ing. en Sistemas \\
 & Informatica 1 \\
 & Prof. Ernesto Rodriguez - \href{mailto:erodriguez@unis.edu.gt}{erodriguez@unis.edu.gt} \\
\end{tabular}
\\\\\\

\begin{center}
        \horrule{0.5pt}
        \huge{Hoja de trabajo \#6} \\
        \large{Fecha de entrega: 6 de Septiembre, 2018 - 11:59pm} \\
        \horrule{1pt}
\end{center}

\emph{Instrucciones: Resolver cada uno de los ejercicios siguiendo sus respectivas
instrucciones. El trabajo debe ser entregado a traves de Github, en su repositorio del curso, colocado en una
carpeta llamada "Hoja de trabajo 6". Al menos que la pregunta indique diferente, todas las
respuestas a preguntas escritas deben presentarse en un documento formato pdf, el cual
haya sido generado mediante Latex. }\\\\

{\bf Nota: }En este deber se omitira la ubicaci\'on exacta del compilador
de elm, y solo se escribira elm. Por ejemplo, en vez de escribir:\\\\
$>\mathtt{node\_modules}\backslash\mathtt{elm\ repl}$\\\\
Se escribira:\\\\
$>\mathtt{elm\ repl}$\\\\
Adicionalmente, asegurarse que las funciones y modulos que sean declarados
en su deber correspondan exactamente a los nombres escritos en dicho deber
ya que se utilizaran pruebas automatizadas para calificar.

\section*{Ejercicio \#1 (25\%)}
Los {\bf numeros naturales unarios} se pueden definir de la siguiente manera:
$$
        \mathbf{type}\ \mathtt{Natural}\ =\ \mathtt{Suc\ Natural}\ |\ \mathtt{Cero}
$$
En donde $\mathtt{Cero}$ corresponde al numero cero (0) y $\mathtt{Suc}$ es
el constructor que construye un natural representandolo como el sucesor de
otro numero. Ejercicio:
\begin{itemize}
        \item{Definir en Elm la funci\'on ``$\mathtt{resta}\ :\ \mathtt{Natural}
        \rightarrow\mathtt{Natural}\rightarrow\mathtt{Natural}$'' la cual calcula
        la resta entre dos naturales como estan definidos anteriormente. Si, el
        resultado fuese a ser negativo, retornar cero.}
        \item{Definir la funci\'on ``$\mathtt{multiplicacion}\ :\ \mathtt{Natural}
        \rightarrow\mathtt{Natural}\rightarrow\mathtt{Natural}$'' la cula multiplica
        dos naturales como fueron definidos anteriormente.}
        \item{Definir la funci\'on ``$\mathtt{division}\ :\ \mathtt{Natural}
        \rightarrow\mathtt{Natural}\rightarrow(\mathtt{Natural},\mathtt{Natural})$'' la cual debe
        calcular la division y el residuo resultante de dividir un natural dentro
        del otro.}
\end{itemize}

\section*{Ejercicio \#2 (25\%)}

Definir el tipo $\mathtt{Expresion}$ para representar {\bf expressiones matematicas} en Elm. Una expresion matematica
esta compuesta de los siguientes casos:
\begin{itemize}
        \item{$\mathtt{Valor}$: El cual debe aceptar un entero ($\mathtt{Int}$) como parametro.}
        \item{$\mathtt{Suma}$: El cual debe aceptar dos expresiones como parametro.}
        \item{$\mathtt{Mult}$: El cual debe aceptar dos expresiones como parametro.}
\end{itemize}
Por ejemplo, la expression ``$3+8*5*2+4$'' seria representada (respetando las reglas de 
procedencia de la suma y multiplicaci\'on) como ``$\mathtt{Suma}\ (\mathtt{Valor}\ 3)\ (\mathtt{Suma}
\ (\mathtt{Mult}\ (\mathtt{Valor}\ 8)\ (\mathtt{Mult}\ (\mathtt{Valor}\ 5)\ (\mathtt{Valor}\ 2)))\ (\mathtt{Valor}\ 4))$''

\section*{Ejercicio \#3 (50\%)}
Definir una funci\'on llamada ``$\mathtt{parsear}\ :\ \mathtt{string}\rightarrow\mathtt{Maybe\ Expresion}$'' que
toma una expresi\'on representada como un $\mathtt{string}$ y produce una expresi\'on respetando la
procedencia de operaciones. Se recomienda seguir los siguientes consejos:
\begin{itemize}
        \item{Generalize el tipo $\mathtt{Expresion}$ creando un nuevo tipo llamado
        $\mathtt{GExpresion}$ el cual acepta un parametro y utiliza ese parametro
        en vez de un $\mathtt{Int}$. Luego puede definir el tipo $mathtt{Expresion}$ asi:\\
        $>\mathbf{type\ alias}\ \mathtt{Expresion}=\mathtt{GExpresion}\ \mathtt{Int}$}
        \item{Crear un tipo nuevo llamado $\mathtt{Estado}$, con dos constructores. Un
        constructor acepta un $\mathtt{Int}$ y el otro un $\mathtt{List Char}$. La idea
        es que el primer caso representa un valor ya producido mientras que la lista
        de caracteres representa un valor que aun necesita procesamiento.}
        \item{El tipo estado se utilizara en las expresiones para facilitar
        la conversion. El algoritmo es el siguiente:
        \begin{enumerate}
                \item{Empezar con un solo $\mathtt{Valor}$, en el cual coloca un
                $\mathtt{Estado}$ con la expresi\'on entera.}
                \item{Buscar en el $\mathtt{Estado}$ el operador de menor
                procedencia. ie. la suma m\'as a la derecha}
                \item{Selecci\'onar el constructor adecuado en base a esa operaci\'on ($\mathtt{Suma}$ o $\mathtt{Mult}$)}
                \item{Construir los valores de ese constructor llamando recursivamente a la funci\'on.}
        \end{enumerate}
        }
\end{itemize}
Para su implementaci\'on solo debe considerar operaci\'ones bien formadas que no contengan espacios. Si la
operaci\'on no cumple estos criterios, puede retornar $\mathtt{Nothing}$

\end{document}