%%%%%%%%%%%%%%%%%%%%%%%%%%%%%%%%%%%%%%%%%
% Programming/Coding Assignment
% LaTeX Template
%
% This template has been downloaded from:
% http://www.latextemplates.com
%
% Original author:
% Ted Pavlic (http://www.tedpavlic.com)
%
% Note:
% The \lipsum[#] commands throughout this template generate dummy text
% to fill the template out. These commands should all be removed when 
% writing assignment content.
%
% This template uses a Perl script as an example snippet of code, most other
% languages are also usable. Configure them in the "CODE INCLUSION 
% CONFIGURATION" section.
%
%%%%%%%%%%%%%%%%%%%%%%%%%%%%%%%%%%%%%%%%%

%----------------------------------------------------------------------------------------
%	PACKAGES AND OTHER DOCUMENT CONFIGURATIONS
%----------------------------------------------------------------------------------------

\documentclass{article}

\usepackage{fancyhdr} % Required for custom headers
\usepackage{lastpage} % Required to determine the last page for the footer
\usepackage{extramarks} % Required for headers and footers
\usepackage[usenames,dvipsnames]{color} % Required for custom colors
\usepackage{graphicx} % Required to insert images
\usepackage{listings} % Required for insertion of code
\usepackage{courier} % Required for the courier font
\usepackage{multirow}
\usepackage{hyperref}
\usepackage{amsmath}
\usepackage{amssymb}

% Margins
\topmargin=-0.45in
\evensidemargin=0in
\oddsidemargin=0in
\textwidth=6.5in
\textheight=9.0in
\headsep=0.25in

\linespread{1.1} % Line spacing

%----------------------------------------------------------------------------------------
%	CODE INCLUSION CONFIGURATION
%----------------------------------------------------------------------------------------

\definecolor{MyDarkGreen}{rgb}{0.0,0.4,0.0} % This is the color used for comments
\lstloadlanguages{c} % Load Perl syntax for listings, for a list of other languages supported see: ftp://ftp.tex.ac.uk/tex-archive/macros/latex/contrib/listings/listings.pdf
\lstset{language=[sharp]c, % Use Perl in this example
        frame=single, % Single frame around code
        basicstyle=\small\ttfamily, % Use small true type font
        keywordstyle=[1]\color{Blue}\bf, % Perl functions bold and blue
        keywordstyle=[2]\color{Purple}, % Perl function arguments purple
        keywordstyle=[3]\color{Blue}\underbar, % Custom functions underlined and blue
        identifierstyle=, % Nothing special about identifiers                                         
        commentstyle=\usefont{T1}{pcr}{m}{sl}\color{MyDarkGreen}\small, % Comments small dark green courier font
        stringstyle=\color{Purple}, % Strings are purple
        showstringspaces=false, % Don't put marks in string spaces
        tabsize=5, % 5 spaces per tab
        %
        % Put standard Perl functions not included in the default language here
        morekeywords={rand},
        %
        % Put Perl function parameters here
        morekeywords=[2]{on, off, interp},
        %
        % Put user defined functions here
        morekeywords=[3]{test},
       	%
        morecomment=[l][\color{Blue}]{...}, % Line continuation (...) like blue comment
        numbers=left, % Line numbers on left
        firstnumber=1, % Line numbers start with line 1
        numberstyle=\tiny\color{Blue}, % Line numbers are blue and small
        stepnumber=5 % Line numbers go in steps of 5
}

\newcommand{\horrule}[1]{\rule{\linewidth}{#1}}

% Creates a new command to include a perl script, the first parameter is the filename of the script (without .pl), the second parameter is the caption
\newcommand{\perlscript}[2]{
\begin{itemize}
\item[]\lstinputlisting[caption=#2,label=#1]{#1.cs}
\end{itemize}
}

\begin{document}

\begin{tabular}{l l}
\multirow{5}{*}{\includegraphics[width=2cm]{../../recursos/logo.png}}
 & Universidad del Istmo de Guatemala \\
 & Facultad de Ingenieria \\
 & Ing. en Sistemas \\
 & Informatica 1 \\
 & Prof. Ernesto Rodriguez - \href{mailto:erodriguez@unis.edu.gt}{erodriguez@unis.edu.gt} \\
\end{tabular}
\\\\\\

\begin{center}
        \horrule{0.5pt}
        \huge{Hoja de trabajo \#7} \\
        \large{Fecha de entrega: 27 de Septiembre, 2018 - 11:59pm} \\
        \horrule{1pt}
\end{center}

\emph{Instrucciones: Resolver cada uno de los ejercicios siguiendo sus respectivas
instrucciones. El trabajo debe ser entregado a traves de Github, en su repositorio del curso, colocado en una
carpeta llamada "Hoja de trabajo 7". Al menos que la pregunta indique diferente, todas las
respuestas a preguntas escritas deben presentarse en un documento formato pdf, el cual
haya sido generado mediante Latex. }\\\\

{\bf Nota: }En este deber se omitira la ubicaci\'on exacta del compilador
de elm, y solo se escribira elm. Por ejemplo, en vez de escribir:\\\\
$>\mathtt{node\_modules}\backslash\mathtt{elm\ repl}$\\\\
Se escribira:\\\\
$>\mathtt{elm\ repl}$\\\\
Adicionalmente, asegurarse que las funciones y modulos que sean declarados
en su deber correspondan exactamente a los nombres escritos en dicho deber
ya que se utilizaran pruebas automatizadas para calificar.
\\\\
Para esta tarea necesitara el tipo de datos ``Expressi\'on'', el cual fue
definido en la tarea \#6.

\section*{Ejercicio \#1 (20\%)}
Defina las siguientes funci\'ones:
\begin{itemize}
        \item{La funci\'on $\mathtt{suma}$ de tipo $\mathtt{Int}\rightarrow\mathtt{Int}
        \rightarrow\mathtt{Int}$ que suma dos numeros enteros.}
        \item{La funci\'on $\mathtt{multiplicacion}$ de tipo $\mathtt{Int}\rightarrow\mathtt{Int}
        \rightarrow\mathtt{Int}$, que multiplica dos numeros enteros.}
\end{itemize}


\section*{Ejercicio \#2 (80\%)}

Un {\bf algebra} o {\bf reglas de evaluaci\'on} son un conjunto de reglas que describen como
se reduce una expresi\'on. En el caso de el tipo $\mathtt{Expresion}$, se necesitan dos reglas:
\begin{itemize}
        \item{Regla para hacer sumas de tipo $\mathtt{Int}\rightarrow\mathtt{Int}
        \rightarrow\mathtt{Int}$}
        \item{Regla para hacer multiplicaciones de tipo $\mathtt{Int}\rightarrow\mathtt{Int}
        \rightarrow\mathtt{Int}$}
\end{itemize}

Defina una funci\'on llamada $\mathtt{evaluar}$ de tipo $\mathtt{((Int\rightarrow Int\rightarrow Int)\times
(\mathtt{Int}\rightarrow\mathtt{Int}\rightarrow\mathtt{Int}))}\rightarrow \mathtt{Expression} \mathtt{Int}$.
El primer parametro de la funci\'on es el algebra y el segundo parametro la {\bf expressi\'on}. Su
funci\'on debe utilizar este algebra para reducir la expressi\'on que se le haya dado.

\end{document}