%%%%%%%%%%%%%%%%%%%%%%%%%%%%%%%%%%%%%%%%%
% Programming/Coding Assignment
% LaTeX Template
%
% This template has been downloaded from:
% http://www.latextemplates.com
%
% Original author:
% Ted Pavlic (http://www.tedpavlic.com)
%
% Note:
% The \lipsum[#] commands throughout this template generate dummy text
% to fill the template out. These commands should all be removed when 
% writing assignment content.
%
% This template uses a Perl script as an example snippet of code, most other
% languages are also usable. Configure them in the "CODE INCLUSION 
% CONFIGURATION" section.
%
%%%%%%%%%%%%%%%%%%%%%%%%%%%%%%%%%%%%%%%%%

%----------------------------------------------------------------------------------------
%	PACKAGES AND OTHER DOCUMENT CONFIGURATIONS
%----------------------------------------------------------------------------------------

\documentclass{article}

\usepackage{fancyhdr} % Required for custom headers
\usepackage{lastpage} % Required to determine the last page for the footer
\usepackage{extramarks} % Required for headers and footers
\usepackage[usenames,dvipsnames]{color} % Required for custom colors
\usepackage{graphicx} % Required to insert images
\usepackage{listings} % Required for insertion of code
\usepackage{courier} % Required for the courier font
\usepackage{multirow}
\usepackage{hyperref}
\usepackage{amsmath}
\usepackage{amssymb}


% Margins
\topmargin=-0.45in
\evensidemargin=0in
\oddsidemargin=0in
\textwidth=6.5in
\textheight=9.0in
\headsep=0.25in

\linespread{1.1} % Line spacing

%----------------------------------------------------------------------------------------
%	CODE INCLUSION CONFIGURATION
%----------------------------------------------------------------------------------------

\definecolor{MyDarkGreen}{rgb}{0.0,0.4,0.0} % This is the color used for comments
\lstloadlanguages{c} % Load Perl syntax for listings, for a list of other languages supported see: ftp://ftp.tex.ac.uk/tex-archive/macros/latex/contrib/listings/listings.pdf
\lstset{language=[sharp]c, % Use Perl in this example
        frame=single, % Single frame around code
        basicstyle=\small\ttfamily, % Use small true type font
        keywordstyle=[1]\color{Blue}\bf, % Perl functions bold and blue
        keywordstyle=[2]\color{Purple}, % Perl function arguments purple
        keywordstyle=[3]\color{Blue}\underbar, % Custom functions underlined and blue
        identifierstyle=, % Nothing special about identifiers                                         
        commentstyle=\usefont{T1}{pcr}{m}{sl}\color{MyDarkGreen}\small, % Comments small dark green courier font
        stringstyle=\color{Purple}, % Strings are purple
        showstringspaces=false, % Don't put marks in string spaces
        tabsize=5, % 5 spaces per tab
        %
        % Put standard Perl functions not included in the default language here
        morekeywords={rand},
        %
        % Put Perl function parameters here
        morekeywords=[2]{on, off, interp},
        %
        % Put user defined functions here
        morekeywords=[3]{test},
       	%
        morecomment=[l][\color{Blue}]{...}, % Line continuation (...) like blue comment
        numbers=left, % Line numbers on left
        firstnumber=1, % Line numbers start with line 1
        numberstyle=\tiny\color{Blue}, % Line numbers are blue and small
        stepnumber=5 % Line numbers go in steps of 5
}

\newcommand{\horrule}[1]{\rule{\linewidth}{#1}}

% Creates a new command to include a perl script, the first parameter is the filename of the script (without .pl), the second parameter is the caption
\newcommand{\perlscript}[2]{
\begin{itemize}
\item[]\lstinputlisting[caption=#2,label=#1]{#1.cs}
\end{itemize}
}

\begin{document}

\begin{tabular}{l l}
\multirow{5}{*}{\includegraphics[width=2cm]{../recursos/logo.png}} & Universidad del Istmo de Guatemala \\
 & Facultad de Ingenieria \\
 & Ing. en Sistemas \\
 & Informatica 1 \\
 & Prof. Ernesto Rodriguez - \href{mailto:erodriguez@unis.edu.gt}{erodriguez@unis.edu.gt} \\
\end{tabular}
\\\\\\

\begin{center}
        \horrule{0.5pt}
        \huge{Proyecto Final: Boxworld} \\
        \large{Fecha de entrega: 01 de Mayo, 2019 - 11:59pm} \\
        \horrule{1pt}
\end{center}

\emph{Instrucciones: Resolver cada uno de los ejercicios siguiendo sus respectivas
instrucciones. El trabajo debe ser entregado a traves de Github, en su repositorio del curso, colocado en una carpeta llamada "Proyecto Final".
Al menos que la pregunta indique diferente, todas las respuestas a preguntas escritas deben presentarse en
un documento formato pdf, el cual haya sido generado mediante Latex. }
\\\\
A continuaci\'on se presenta el proyecto final. El objetivo de dicho proyecto es poner
en practica lo aprendido durante el curso para crear un juego interactivo. El objetivo
es que usted cree una versi\'on propia y simplificada del juego ``Boxworld''. Puede
encontrar un ejemplo del juego en \url{http://wallofgame.com/free-online-games/arcade/921/Boxworld.html}.
\section*{Informaci\'on Administrativa}
El proyecto se puede elaborar en grupos de un maximo de 3 integrantes. Por favor ingresar
a \url{https://goo.gl/forms/XpUXU6mkl9qGqDIC2} y llenar el formulario de participaci\'on.
Solamente es necesario un formulario por grupo.
\\\\
El proyecto tiene un valor total de 30\% del curso de los cuales un 10\% se le abonaran
a la zona y un 20\% como la parte practica del examen final. Adicionalmente, puede
obtener hasta un 10\% extra por entregas que excedan el minimo.

\section*{Entrega}
Debe crear un programa que implemente {\bf fielmente} la mecanica del juego ``Boxworld''.
Su programa debe incluir al menos un nivel el cual se debe poder jugar y completar.
Solamente es necesaria una interfaz de consola, lo cual significa que puede utilizar
caracteres arbitrarios para dibujar el mapa, personaje, cajas y sitios donde se debe
colocar la caja. Su programa debe implementar las reglas del juego tal y como las
implementa el juego ejemplar.
\\\\
Se evaluara el uso de buenas practicas de desarrollo de software, esto incluye:
\begin{itemize}
        \item{Utilizaci\'on de clases y objetos para abstraer y modelar el programa}
        \item{Manejo adecuado de memoria}
        \item{Validaci\'on apropiada de indices al utilizar arreglos}
        \item{Encapsulaci\'on correcta de el estado de la clase}
        \item{Utilizar tipos enumerados para expresar el significado de estados}
\end{itemize}
El proyecto debe ser entregado a traves de Git, en el repositorio que usted
indique en el formulario. Adicionalmente habra una presentaci\'on oral en grupo
en la cual el programa debe ser explicado por los integrantes y habran preguntas
directas a cada uno de los integrantes acerca del codigo. Las preguntas pueden
ser de {\bf cualquier parte del codigo} y seran dirigidas a una persona en
perticular. A pesar de ser un trabajo en grupo, la nota final sera individual
ya que estara ponderada por su conocimiento individual del codigo.
\section*{Extras}
Tiene la opci\'on de agregar funcionalidad extra a cambio de un mejor punteo.
Esto incluye (pero no se limita a):
\begin{itemize}
        \item{Utilizar una biblioteca grafica en vez de la consola (como \href{https://www.qt.io/}{QT}).}
        \item{Multiples niveles}
        \item{Cargar niveles desde archivos almacenados en el ordenador}
        \item{Utilizar el lenguaje \href{https://www.rust-lang.org/}{Rust} en vez de C++}
        \item{Musica de fondo mientras se juega}
        \item{Utilizaci\'on de bibliotecas (como \href{https://www.boost.org/}{Boost})
        para apoyar el processo de desarrollo.}
        \item{Utilizaci\'on correcta de punteros inteligentes (\url{https://en.cppreference.com/w/cpp/memory})}
\end{itemize}

\section*{Proyectos Alternativos}
Si lo desea, es possible elaborar otro programa en vez de ``Boxworld''. Por favor
hablarme personalmente si desea hacer esto.

\section*{Honestidad y \'Etica}
El trabajo es grupal y debe ser realizado en su totalidad por los integrantes del
grupo. Si este proyecto es copiado o es programado por otras personas que no son
integrantes del grupo, se otorgara 0\% como nota y sera reportado a la Facultad
de Inegineria.

\end{document}