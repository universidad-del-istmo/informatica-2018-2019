%%%%%%%%%%%%%%%%%%%%%%%%%%%%%%%%%%%%%%%%%
% Programming/Coding Assignment
% LaTeX Template
%
% This template has been downloaded from:
% http://www.latextemplates.com
%
% Original author:
% Ted Pavlic (http://www.tedpavlic.com)
%
% Note:
% The \lipsum[#] commands throughout this template generate dummy text
% to fill the template out. These commands should all be removed when 
% writing assignment content.
%
% This template uses a Perl script as an example snippet of code, most other
% languages are also usable. Configure them in the "CODE INCLUSION 
% CONFIGURATION" section.
%
%%%%%%%%%%%%%%%%%%%%%%%%%%%%%%%%%%%%%%%%%

%----------------------------------------------------------------------------------------
%	PACKAGES AND OTHER DOCUMENT CONFIGURATIONS
%----------------------------------------------------------------------------------------

\documentclass{article}

\usepackage{fancyhdr} % Required for custom headers
\usepackage{lastpage} % Required to determine the last page for the footer
\usepackage{extramarks} % Required for headers and footers
\usepackage[usenames,dvipsnames]{color} % Required for custom colors
\usepackage{graphicx} % Required to insert images
\usepackage{listings} % Required for insertion of code
\usepackage{courier} % Required for the courier font
\usepackage{multirow}
\usepackage{hyperref}
\usepackage[utf8]{inputenc}

% Margins
\topmargin=-0.45in
\evensidemargin=0in
\oddsidemargin=0in
\textwidth=6.5in
\textheight=9.0in
\headsep=0.25in

\linespread{1.1} % Line spacing

%----------------------------------------------------------------------------------------
%	CODE INCLUSION CONFIGURATION
%----------------------------------------------------------------------------------------

\definecolor{MyDarkGreen}{rgb}{0.0,0.4,0.0} % This is the color used for comments
\lstloadlanguages{c} % Load Perl syntax for listings, for a list of other languages supported see: ftp://ftp.tex.ac.uk/tex-archive/macros/latex/contrib/listings/listings.pdf
\lstset{language=[sharp]c, % Use Perl in this example
        frame=single, % Single frame around code
        basicstyle=\small\ttfamily, % Use small true type font
        keywordstyle=[1]\color{Blue}\bf, % Perl functions bold and blue
        keywordstyle=[2]\color{Purple}, % Perl function arguments purple
        keywordstyle=[3]\color{Blue}\underbar, % Custom functions underlined and blue
        identifierstyle=, % Nothing special about identifiers                                         
        commentstyle=\usefont{T1}{pcr}{m}{sl}\color{MyDarkGreen}\small, % Comments small dark green courier font
        stringstyle=\color{Purple}, % Strings are purple
        showstringspaces=false, % Don't put marks in string spaces
        tabsize=5, % 5 spaces per tab
        %
        % Put standard Perl functions not included in the default language here
        morekeywords={rand},
        %
        % Put Perl function parameters here
        morekeywords=[2]{on, off, interp},
        %
        % Put user defined functions here
        morekeywords=[3]{test},
       	%
        morecomment=[l][\color{Blue}]{...}, % Line continuation (...) like blue comment
        numbers=left, % Line numbers on left
        firstnumber=1, % Line numbers start with line 1
        numberstyle=\tiny\color{Blue}, % Line numbers are blue and small
        stepnumber=5 % Line numbers go in steps of 5
}

\newcommand{\horrule}[1]{\rule{\linewidth}{#1}}

% Creates a new command to include a perl script, the first parameter is the filename of the script (without .pl), the second parameter is the caption
\newcommand{\perlscript}[2]{
\begin{itemize}
\item[]\lstinputlisting[caption=#2,label=#1]{#1.cs}
\end{itemize}
}

\begin{document}

\begin{tabular}{l l}
\multirow{5}{*}{\includegraphics[width=2cm]{../../Recursos/logo.png}} & Universidad del Istmo de Guatemala \\
 & Facultad de Ingenieria \\
 & Ing. en Sistemas \\
 & Informatica 1 \\
 & Prof. Ernesto Rodriguez - \href{mailto:erodriguez@unis.edu.gt}{erodriguez@unis.edu.gt} \\
\end{tabular}
\\\\\\

\begin{center}
        \horrule{0.5pt}
        \huge{Examen Parcial \#1} \\
        \large{Tiempo de resoluci\'on: 90 minutos} \\
        \horrule{1pt}
\end{center}

\emph{Instrucciones: Responder las preguntas que se presentan a continuaci\'on y hacer los ejercicios que se presenten
        a continuaci\'on.}


\section{Pregunta \#1 (20\%)}
El famoso matematico Euler hizo la siguiente pregunta: ¿Es possible curzar todos los puentes
de K\"onigsberg sin pasar dos veces por el mismo puente? A continuaci\'on se muestra un
mapa de los puentes de K\"onigsberg:
\begin{center}
\includegraphics[width=8cm]{bridges.png}
\end{center}
Su tarea es crear un grafo a partir de estos puentes. Para ello debe:
\begin{itemize}
        \item{Definir el conjunto de nodos}
        \item{Definir el conjunto de vertices}
\end{itemize}
La intenci\'on es utilizar dicho grafo para buscar soluciones al
problema, por lo cual su selecci\'on de nodos y vertices se debe
adecuar al problema. No es necesario encontrar una respuesta.
\section*{Pregunta \#2 (20\%)}
Demostrar utilizando inducci\'on que la formula de Gauss para sumatorias es correcta:
\[
        \sum_{i=1}^{n}{i}=\frac{n(n+1)}{2}
\]
donde $\sum_{i=1}^{n}i=1+2+3+4+\ \ldots\ +n$.
\\\\
Para esta demostraci\'on, su caso base debe ser
$n=1$ en vez de $n=0$. Sin embargo, la demostraci\'on
del caso inductivo procede de la misma forma que
se ha estudiado en clase.

\section*{Pregunta \#3 (20\%)}
Definir inductivamente la funcion $\sum(n)$ para numeros naturales unarios la cual tiene
el efecto de calcular la suma de $1$ hasta $n$. En otras palabras:
\[
        \sum(n)=1+2+3+4+\ \ldots\ +n
\]
Puede apoyarse de la suma $\oplus$ de numeros naturales unarios para su definici\'on:
\[
        a\oplus b =
                \left\{
                        \begin{array}{ll}
                                b  & \mbox{si } a = 0 \\
                                s(i\oplus b) & \mbox{si } a = s(i)
                        \end{array}
                \right.
\]

\section*{Pregutna \# 4 (20\%)}
Demostrar por medio de inducci\'on la comutatividad de la suma de
numeros naturales unarios: $a\oplus b = b\oplus a$

\section*{Pregunta \#5 (20\%)}
Dada la funci\'on $a\geq b$ para numeros naturales unarios:
\[
        a\geq b =
                \left\{
                        \begin{array}{ll}
                                s(o)  & \mbox{si } b = o \\
                                o & \mbox{si } a = o \\
                                i\geq j & \mbox{si } a = s(i)\ \&\ b = s(j)
                        \end{array}
                \right.
\]
Demostrar utilizando inducci\'on que $((n\oplus n)\geq n) = s(o)$. Puede
hacer uso de la asociatividad y comutabilidad de la suma de numeros
unarios para su demostraci\'on.

\end{document}