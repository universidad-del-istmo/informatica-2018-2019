%%%%%%%%%%%%%%%%%%%%%%%%%%%%%%%%%%%%%%%%%
% Programming/Coding Assignment
% LaTeX Template
%
% This template has been downloaded from:
% http://www.latextemplates.com
%
% Original author:
% Ted Pavlic (http://www.tedpavlic.com)
%
% Note:
% The \lipsum[#] commands throughout this template generate dummy text
% to fill the template out. These commands should all be removed when 
% writing assignment content.
%
% This template uses a Perl script as an example snippet of code, most other
% languages are also usable. Configure them in the "CODE INCLUSION 
% CONFIGURATION" section.
%
%%%%%%%%%%%%%%%%%%%%%%%%%%%%%%%%%%%%%%%%%

%----------------------------------------------------------------------------------------
%	PACKAGES AND OTHER DOCUMENT CONFIGURATIONS
%----------------------------------------------------------------------------------------

\documentclass{article}

\usepackage{fancyhdr} % Required for custom headers
\usepackage{lastpage} % Required to determine the last page for the footer
\usepackage{extramarks} % Required for headers and footers
\usepackage[usenames,dvipsnames]{color} % Required for custom colors
\usepackage{graphicx} % Required to insert images
\usepackage{listings} % Required for insertion of code
\usepackage{courier} % Required for the courier font
\usepackage{multirow}
\usepackage{hyperref}

% Margins
\topmargin=-0.45in
\evensidemargin=0in
\oddsidemargin=0in
\textwidth=6.5in
\textheight=9.0in
\headsep=0.25in

\linespread{1.1} % Line spacing

%----------------------------------------------------------------------------------------
%	CODE INCLUSION CONFIGURATION
%----------------------------------------------------------------------------------------

\definecolor{MyDarkGreen}{rgb}{0.0,0.4,0.0} % This is the color used for comments
\lstloadlanguages{c} % Load Perl syntax for listings, for a list of other languages supported see: ftp://ftp.tex.ac.uk/tex-archive/macros/latex/contrib/listings/listings.pdf
\lstset{language=[sharp]c, % Use Perl in this example
        frame=single, % Single frame around code
        basicstyle=\small\ttfamily, % Use small true type font
        keywordstyle=[1]\color{Blue}\bf, % Perl functions bold and blue
        keywordstyle=[2]\color{Purple}, % Perl function arguments purple
        keywordstyle=[3]\color{Blue}\underbar, % Custom functions underlined and blue
        identifierstyle=, % Nothing special about identifiers                                         
        commentstyle=\usefont{T1}{pcr}{m}{sl}\color{MyDarkGreen}\small, % Comments small dark green courier font
        stringstyle=\color{Purple}, % Strings are purple
        showstringspaces=false, % Don't put marks in string spaces
        tabsize=5, % 5 spaces per tab
        %
        % Put standard Perl functions not included in the default language here
        morekeywords={rand},
        %
        % Put Perl function parameters here
        morekeywords=[2]{on, off, interp},
        %
        % Put user defined functions here
        morekeywords=[3]{test},
       	%
        morecomment=[l][\color{Blue}]{...}, % Line continuation (...) like blue comment
        numbers=left, % Line numbers on left
        firstnumber=1, % Line numbers start with line 1
        numberstyle=\tiny\color{Blue}, % Line numbers are blue and small
        stepnumber=5 % Line numbers go in steps of 5
}

\newcommand{\horrule}[1]{\rule{\linewidth}{#1}}

\newcommand\doubleplus{\ensuremath{\mathbin{+\mkern-10mu+}}}

% Creates a new command to include a perl script, the first parameter is the filename of the script (without .pl), the second parameter is the caption
\newcommand{\perlscript}[2]{
\begin{itemize}
\item[]\lstinputlisting[caption=#2,label=#1]{#1.cs}
\end{itemize}
}

\begin{document}

\begin{tabular}{l l}
\multirow{5}{*}{\includegraphics[width=2cm]{../../recursos/logo.png}} & Universidad del Istmo de Guatemala \\
 & Facultad de Ingenieria \\
 & Ing. en Sistemas \\
 & Informatica II \\
 & Prof. Ernesto Rodriguez - \href{mailto:erodriguez@unis.edu.gt}{erodriguez@unis.edu.gt} \\
\end{tabular}
\\\\\\

\begin{center}
        \horrule{0.5pt}
        \huge{Laboratorio \#7} \\
        \large{Fecha de entrega: 21 de Marzo, 2019 - 11:59pm} \\
        \horrule{1pt}
\end{center}

\emph{Instrucciones: Resolver cada uno de los ejercicios siguiendo sus respectivas
instrucciones. El trabajo debe ser entregado a traves de Github, en su repositorio del curso, colocado en una carpeta llamada "Laboratorio \#7".
Al menos que la pregunta indique diferente, todas las respuestas a preguntas escritas deben presentarse en
un documento formato pdf, el cual haya sido generado mediante Latex. Este laboratorio
debe ser elaborado en parejas.}

\section*{Tarea \#1 (20\%)}

Defina una {\bf clase virtual} llamada ``Figura''. Esta clase debe
tener 3 {\bf metodos virtuales}:
\begin{itemize}
        \item{``$\mathtt{double}\ area()\ \mathtt{const}$'' el cual debe calcular el area de la figura}
        \item{``$\mathtt{void}\ escalar(\mathtt{const\ double}\ escala)$'' el cual debe cambiar el tama\~no de la
        figura proporcionalmente al parametro dado}
        \item{``$\mathtt{void}\ mover(\mathtt{const}\ Vector2d\ lugar)$'' el cual debe desplazar la figura en la direcci\'on
        indicada por el vector que se dio como parametro.}
        \item{``$\mathtt{bool}\ estaAdentro(\mathtt{const}\ Vector2d\ posicion)\ \mathtt{const}$'', el cual acepta un
        vector y retorna \texttt{true} si el vector esta adentro o \texttt{false} de lo contrario}
        \item{``$\mathtt{string}\ toString()\ \mathtt{const}$'', el cual debe retornar una representaci\'on
        de la figura.}
\end{itemize}

\section*{Tarea \#2 (20\%)}
Definir la clase ``Cicrulo'' la cual debe implementar la {\bf clase virtual} ``Figura''. Para representar
el circulo debe utilizar un ``Vector2d'' el cual representa la posici\'on del centro del circulo
y un \texttt{double} para representar el radio del circulo. Debe tener un constructor que acepta
el centro y radio como parametro. Utilizar el analisis matematico necesario para implementar
los metodos de ``Figura''. Recuerde que el area de un circlo esta dado por $A_{\circ}=\pi*r^2$. El
metodo ``toString'' debe representar el circulo en el siguiente formato: ``c(x,y,r)''.

\section*{Tarea \#3 (20\%)}
Definir la clase ``Rectangulo'' la cual debe implementar la {\bf clase virtual} ``Figura''. Para
representar el rectangulo se debe utilizar un ``Vector2d'' para indicar la posici\'on de la
esquina inferior del rectangulo, un \texttt{double} para indicar el ancho y otro \texttt{double}
para indicar el largo del rectangulo. El metodo ``toString'' debe representar el rectangulo
en el siguiente formato: ``r(x,y,largo,ancho)''.

\section*{Tarea \#4 (20\%)}
Definir la funci\'on ``$\mathtt{const}\ abarcar(Figura\&\ figura,\ Vector2d[]\ puntos,\ \mathtt{int}\ cantidad)$''
la cual acepta una Figura como parametro y un arreglo con ``cantidad'' vectores de dos dimensiones. Este metodo
debe mover y/o escalar la figura de tal manera que todos los puntos esten adentro de la Figura. Este metodo
debe intentar que la figura tenga al menor area possible despues de la ejecuci\'on, siempre y cuando
todos los puntos se encuentren dentro de la nueva figura.

\section*{Tarea \#5 (20\%)}
Declarar un metodo ``main'' el cual debe hacer lo siguiente:
\begin{enumerate}
        \item{Solicitar al usuario un caracter, estos pueden ser ``v'' para vector, ``c'' para circulo
        o ``r'' para rectangulo.}
        \item{Si el caracter es ``v'':
                \begin{enumerate}
                        \item{Solicitar dos numeros reales que corresponden a las coordenadas
                        del vector}
                        \item{Regresar al paso \#1}
                \end{enumerate}
        }
        \item{Si el caracter es ``c'':
                \begin{enumerate}
                        \item{Solicitar dos numeros reales correspondiendo al vector del centro del circulo}
                        \item{Solicitar un numero real correspondiente al radio del circulo}
                        \item{Ir al paso \#5}
                \end{enumerate}
        }
        \item{Si el caracter es ``r'':
                \begin{enumerate}
                        \item{Solicitar dos numeros reales correspondiendo al vector de la esquina inferior del
                        rectangulo}
                        \item{Solicitar un numero real correspondiendo al ancho del rectangulo}
                        \item{Solicitar un numero real correspondiendo al largo del rectangulo}
                        \item{Ir al paso \#5}
                \end{enumerate}
        }
        \item{Llamar a la funci\'on ``abarcar'' con los vectores y figura que fueron ingresados en los pasos 1-4.}
        \item{Imprimir la figura resultante y terminar}
\end{enumerate}

\end{document}