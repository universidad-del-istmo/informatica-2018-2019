%%%%%%%%%%%%%%%%%%%%%%%%%%%%%%%%%%%%%%%%%
% Programming/Coding Assignment
% LaTeX Template
%
% This template has been downloaded from:
% http://www.latextemplates.com
%
% Original author:
% Ted Pavlic (http://www.tedpavlic.com)
%
% Note:
% The \lipsum[#] commands throughout this template generate dummy text
% to fill the template out. These commands should all be removed when 
% writing assignment content.
%
% This template uses a Perl script as an example snippet of code, most other
% languages are also usable. Configure them in the "CODE INCLUSION 
% CONFIGURATION" section.
%
%%%%%%%%%%%%%%%%%%%%%%%%%%%%%%%%%%%%%%%%%

%----------------------------------------------------------------------------------------
%	PACKAGES AND OTHER DOCUMENT CONFIGURATIONS
%----------------------------------------------------------------------------------------

\documentclass{article}

\usepackage{fancyhdr} % Required for custom headers
\usepackage{lastpage} % Required to determine the last page for the footer
\usepackage{extramarks} % Required for headers and footers
\usepackage[usenames,dvipsnames]{color} % Required for custom colors
\usepackage{graphicx} % Required to insert images
\usepackage{listings} % Required for insertion of code
\usepackage{courier} % Required for the courier font
\usepackage{multirow}
\usepackage{hyperref}

% Margins
\topmargin=-0.45in
\evensidemargin=0in
\oddsidemargin=0in
\textwidth=6.5in
\textheight=9.0in
\headsep=0.25in

\linespread{1.1} % Line spacing

%----------------------------------------------------------------------------------------
%	CODE INCLUSION CONFIGURATION
%----------------------------------------------------------------------------------------

\definecolor{MyDarkGreen}{rgb}{0.0,0.4,0.0} % This is the color used for comments
\lstloadlanguages{c} % Load Perl syntax for listings, for a list of other languages supported see: ftp://ftp.tex.ac.uk/tex-archive/macros/latex/contrib/listings/listings.pdf
\lstset{language=[sharp]c, % Use Perl in this example
        frame=single, % Single frame around code
        basicstyle=\small\ttfamily, % Use small true type font
        keywordstyle=[1]\color{Blue}\bf, % Perl functions bold and blue
        keywordstyle=[2]\color{Purple}, % Perl function arguments purple
        keywordstyle=[3]\color{Blue}\underbar, % Custom functions underlined and blue
        identifierstyle=, % Nothing special about identifiers                                         
        commentstyle=\usefont{T1}{pcr}{m}{sl}\color{MyDarkGreen}\small, % Comments small dark green courier font
        stringstyle=\color{Purple}, % Strings are purple
        showstringspaces=false, % Don't put marks in string spaces
        tabsize=5, % 5 spaces per tab
        %
        % Put standard Perl functions not included in the default language here
        morekeywords={rand},
        %
        % Put Perl function parameters here
        morekeywords=[2]{on, off, interp},
        %
        % Put user defined functions here
        morekeywords=[3]{test},
       	%
        morecomment=[l][\color{Blue}]{...}, % Line continuation (...) like blue comment
        numbers=left, % Line numbers on left
        firstnumber=1, % Line numbers start with line 1
        numberstyle=\tiny\color{Blue}, % Line numbers are blue and small
        stepnumber=5 % Line numbers go in steps of 5
}

\newcommand{\horrule}[1]{\rule{\linewidth}{#1}}

\newcommand\doubleplus{\ensuremath{\mathbin{+\mkern-10mu+}}}

% Creates a new command to include a perl script, the first parameter is the filename of the script (without .pl), the second parameter is the caption
\newcommand{\perlscript}[2]{
\begin{itemize}
\item[]\lstinputlisting[caption=#2,label=#1]{#1}
\end{itemize}
}

\begin{document}

\begin{tabular}{l l}
\multirow{5}{*}{\includegraphics[width=2cm]{../../recursos/logo.png}} & Universidad del Istmo de Guatemala \\
 & Facultad de Ingenieria \\
 & Ing. en Sistemas \\
 & Informatica II \\
 & Prof. Ernesto Rodriguez - \href{mailto:erodriguez@unis.edu.gt}{erodriguez@unis.edu.gt} \\
\end{tabular}
\\\\\\

\begin{center}
        \horrule{0.5pt}
        \huge{Laboratorio \#9} \\
        \large{Fecha de entrega: 11 de Abril, 2019 - 11:59pm} \\
        \horrule{1pt}
\end{center}

\emph{Instrucciones: Resolver cada uno de los ejercicios siguiendo sus respectivas
instrucciones. El trabajo debe ser entregado a traves de Github, en su repositorio del curso, colocado en una carpeta llamada "Laboratorio \#9".
Al menos que la pregunta indique diferente, todas las respuestas a preguntas escritas deben presentarse en
un documento formato pdf, el cual haya sido generado mediante Latex. Este laboratorio
debe ser elaborado en parejas.}

\section*{Tarea \#1 (20\%)}

Declare una \emph{clase virtual} llamada ``Expression''. Esta clase debe tener el metodo
virtual ``$\mathtt{double}\ evaluar()$''.

\section*{Tarea \#2 (20\%)}

Declare la clase ``Valor'' la cual debe heredar de la clase ``Expression''. Esta clase
debe:
\begin{itemize}
        \item{Tener un constructor que acepta un \texttt{double} llamado ``valor''}
        \item{Su metodo ``evaluar'' debe retornar el ``valor'' que se le dio a su constructor}
\end{itemize}
Esta clase pretende representar las expressiones numericas.
                        
\section*{Tarea \#3 (20\%)}

Declare una \emph{clase virtual} llamada ``OperacionBinaria''. Esta clase debe:
\begin{itemize}
        \item{Tener un constructor que acepta dos instancias de $Expression*$ llamadas
        ``operador1'' y ``operador2''}
        \item{Tiene un metodo virtual ``$\mathtt{double}\ operar(\mathtt{const\ double}\ op1,\ \mathtt{const\ double}\ op2)\ \mathtt{const}$''}
        \item{Su metodo ``evaluar'' debe:
                \begin{enumerate}
                        \item{Llamar al metodo ``evaluar'' de ``operador1'' y ``operador2''}
                        \item{Llamar a su metodo ``operar'' con los resultados del paso anterior}
                        \item{Retornar el resultado de llamar al metodo ``operar''}
                \end{enumerate} 
        }
\end{itemize}

\section*{Tarea \#4 (20\%)}
Declare las clases ``Suma'' y ``Multiplicacion''. Estas clases deben heredar de
la clase ``OperacionBinaria''. El metodo ``operar'' de la clase ``Suma'' debe
sumar sus parametros y el metodo ``operar'' de la clase ``Multiplicacion'' debe
multiplicar sus parametros.

\section*{Tarea \#5 (20\%)}
Defina una funci\'on ``$\mathtt{bool}\ parse(\mathtt{const\ std::string}\ expression,\ Expression*\ resultado)$'' la
cual debe leer el string que se le dio como parametro y construir una instancia de la clase ``Expresion''
a partir de ese string. Por ejemplo:

\perlscript{parse.cc}{}

\end{document}