%%%%%%%%%%%%%%%%%%%%%%%%%%%%%%%%%%%%%%%%%
% Programming/Coding Assignment
% LaTeX Template
%
% This template has been downloaded from:
% http://www.latextemplates.com
%
% Original author:
% Ted Pavlic (http://www.tedpavlic.com)
%
% Note:
% The \lipsum[#] commands throughout this template generate dummy text
% to fill the template out. These commands should all be removed when 
% writing assignment content.
%
% This template uses a Perl script as an example snippet of code, most other
% languages are also usable. Configure them in the "CODE INCLUSION 
% CONFIGURATION" section.
%
%%%%%%%%%%%%%%%%%%%%%%%%%%%%%%%%%%%%%%%%%

%----------------------------------------------------------------------------------------
%	PACKAGES AND OTHER DOCUMENT CONFIGURATIONS
%----------------------------------------------------------------------------------------

\documentclass{article}

\usepackage{fancyhdr} % Required for custom headers
\usepackage{lastpage} % Required to determine the last page for the footer
\usepackage{extramarks} % Required for headers and footers
\usepackage[usenames,dvipsnames]{color} % Required for custom colors
\usepackage{graphicx} % Required to insert images
\usepackage{listings} % Required for insertion of code
\usepackage{courier} % Required for the courier font
\usepackage{multirow}
\usepackage{hyperref}


% Margins
\topmargin=-0.45in
\evensidemargin=0in
\oddsidemargin=0in
\textwidth=6.5in
\textheight=9.0in
\headsep=0.25in

\linespread{1.1} % Line spacing

%----------------------------------------------------------------------------------------
%	CODE INCLUSION CONFIGURATION
%----------------------------------------------------------------------------------------

\definecolor{MyDarkGreen}{rgb}{0.0,0.4,0.0} % This is the color used for comments
\lstloadlanguages{c} % Load Perl syntax for listings, for a list of other languages supported see: ftp://ftp.tex.ac.uk/tex-archive/macros/latex/contrib/listings/listings.pdf
\lstset{language=[sharp]c, % Use Perl in this example
        frame=single, % Single frame around code
        basicstyle=\small\ttfamily, % Use small true type font
        keywordstyle=[1]\color{Blue}\bf, % Perl functions bold and blue
        keywordstyle=[2]\color{Purple}, % Perl function arguments purple
        keywordstyle=[3]\color{Blue}\underbar, % Custom functions underlined and blue
        identifierstyle=, % Nothing special about identifiers                                         
        commentstyle=\usefont{T1}{pcr}{m}{sl}\color{MyDarkGreen}\small, % Comments small dark green courier font
        stringstyle=\color{Purple}, % Strings are purple
        showstringspaces=false, % Don't put marks in string spaces
        tabsize=5, % 5 spaces per tab
        %
        % Put standard Perl functions not included in the default language here
        morekeywords={rand},
        %
        % Put Perl function parameters here
        morekeywords=[2]{on, off, interp},
        %
        % Put user defined functions here
        morekeywords=[3]{test},
       	%
        morecomment=[l][\color{Blue}]{...}, % Line continuation (...) like blue comment
        numbers=left, % Line numbers on left
        firstnumber=1, % Line numbers start with line 1
        numberstyle=\tiny\color{Blue}, % Line numbers are blue and small
        stepnumber=5 % Line numbers go in steps of 5
}

\newcommand{\horrule}[1]{\rule{\linewidth}{#1}}

\newcommand\doubleplus{\ensuremath{\mathbin{+\mkern-10mu+}}}

% Creates a new command to include a perl script, the first parameter is the filename of the script (without .pl), the second parameter is the caption
\newcommand{\perlscript}[2]{
\begin{itemize}
\item[]\lstinputlisting[caption=#2,label=#1]{#1.cs}
\end{itemize}
}

\begin{document}

\begin{tabular}{l l}
\multirow{5}{*}{\includegraphics[width=2cm]{../../recursos/logo.png}} & Universidad del Istmo de Guatemala \\
 & Facultad de Ingenieria \\
 & Ing. en Sistemas \\
 & Informatica II \\
 & Prof. Ernesto Rodriguez - \href{mailto:erodriguez@unis.edu.gt}{erodriguez@unis.edu.gt} \\
\end{tabular}
\\\\\\

\begin{center}
        \horrule{0.5pt}
        \huge{Laboratorio \#4} \\
        \large{Fecha de entrega: 21 de Febrero, 2019 - 11:59pm} \\
        \horrule{1pt}
\end{center}

\emph{Instrucciones: Resolver cada uno de los ejercicios siguiendo sus respectivas
instrucciones. El trabajo debe ser entregado a traves de Github, en su repositorio del curso, colocado en una carpeta llamada "Laboratorio \#3".
Al menos que la pregunta indique diferente, todas las respuestas a preguntas escritas deben presentarse en
un documento formato pdf, el cual haya sido generado mediante Latex. Este laboratorio
debe ser elaborado en parejas.}

\section*{Tarea \#1 (25\%)}

En C++ existen dos maneras de pasar parametros a un metodo: \emph{por valor}
o \emph{por referencia}. En el codigo adjunto (``Tarea1.cc'') hay una funcion
llamada ``porValor'' y otra llamada ``porReferencia'' que utilizan estas formas
de pasar parametros. Explique mediante comentarios en el codigo por que al
final de la ejecuci\'on del programa, la variable ``valor1'' permanece
con el valor 0 mientras que la variable ``valor2'' tiene valor de 42.

\section*{Tarea \#2 (25\%)}

Un \emph{vector espacial} representa un punto en un espacio fisico. Dicho
vector se puede representar como una \emph{lista ordenada} de numeros. Nosotros
utilizaremos arreglos para representar dichos vectores. La cantidad de entradas
que tiene un vector se le conoce como la \emph{dimensionalidad del vector}, tambien
llamado ``numero de dimensiones``. Por ejemplo, el vector $\langle 4,5,3 \rangle$
es un vector de 3 dimensiones que se representaria con un arreglo $\mathtt{int}\ x=\{4,5,3\};$.
\\\\
El \emph{produto punto} entre dos vectores esta definido asi: $\bar x\cdot \bar y:=\sum_{i=0}^{dims}\bar x_i*\bar y_i$. Defina una funci\'on \\``$\mathtt{float}\ productoPunto(\mathtt{const\ float*}\ x,\ \mathtt{const\ float*}\ y,\ \mathtt{const\ int}\ dims)$'' que
calcule el producto punto entre dos vectores de dimensiones indicadas por el
parametro $dims$.

\section*{Tarea \#3 (25\%)}
Si el producto punto entre dos vectores es cero (0). Se dice que los vectores
son \emph{ortogonales}. Defina la funci\'on ``$\mathtt{bool}\ sonOrtogonales(\mathtt{const\ float*}\ x,\ \mathtt{const\ float*}\ y,\ \mathtt{int}\ dims)$ que determine si dos vectores son ortogonales o no. Esta funci\'on debe
considerar dos vectores ortogonales si el producto punto es cero redondeado a
{\bf dos posiciones decimales}.

\section*{Tarea \#4 (25\%)}

Un conjunto de vectores $\mathbf{V}$ se considera una ``base de espacio vectorial''
si:
\begin{enumerate}
    \item{El numero de vectores en $\mathbf{V}$ es el mismo que la dimension
    de los vectores.}
    \item{Todos los vectores de $\mathbf{V}$ son ortogonales entre si}
\end{enumerate}
Defina la funci\'on ``$\mathtt{bool}\ esBase(\mathtt{const\ float**}\ vectores,\ \mathtt{const\ int}\ dims)$'' la cual determina si un conjunto de vectores de
dimension ``$dims$'' es una base de un espacio vectorial. Pede hacer uso de la
propiedad $\bar x\cdot\bar y=0\wedge\ \bar y\cdot\bar z=0\Rightarrow \bar x\cdot\bar z = 0$ para que su codigo sea mas compacto.

\end{document}