%%%%%%%%%%%%%%%%%%%%%%%%%%%%%%%%%%%%%%%%%
% Programming/Coding Assignment
% LaTeX Template
%
% This template has been downloaded from:
% http://www.latextemplates.com
%
% Original author:
% Ted Pavlic (http://www.tedpavlic.com)
%
% Note:
% The \lipsum[#] commands throughout this template generate dummy text
% to fill the template out. These commands should all be removed when 
% writing assignment content.
%
% This template uses a Perl script as an example snippet of code, most other
% languages are also usable. Configure them in the "CODE INCLUSION 
% CONFIGURATION" section.
%
%%%%%%%%%%%%%%%%%%%%%%%%%%%%%%%%%%%%%%%%%

%----------------------------------------------------------------------------------------
%	PACKAGES AND OTHER DOCUMENT CONFIGURATIONS
%----------------------------------------------------------------------------------------

\documentclass{article}

\usepackage{fancyhdr} % Required for custom headers
\usepackage{lastpage} % Required to determine the last page for the footer
\usepackage{extramarks} % Required for headers and footers
\usepackage[usenames,dvipsnames]{color} % Required for custom colors
\usepackage{graphicx} % Required to insert images
\usepackage{listings} % Required for insertion of code
\usepackage{courier} % Required for the courier font
\usepackage{multirow}
\usepackage{hyperref}

% Margins
\topmargin=-0.45in
\evensidemargin=0in
\oddsidemargin=0in
\textwidth=6.5in
\textheight=9.0in
\headsep=0.25in

\linespread{1.1} % Line spacing

%----------------------------------------------------------------------------------------
%	CODE INCLUSION CONFIGURATION
%----------------------------------------------------------------------------------------

\definecolor{MyDarkGreen}{rgb}{0.0,0.4,0.0} % This is the color used for comments
\lstloadlanguages{c} % Load Perl syntax for listings, for a list of other languages supported see: ftp://ftp.tex.ac.uk/tex-archive/macros/latex/contrib/listings/listings.pdf
\lstset{language=[sharp]c, % Use Perl in this example
        frame=single, % Single frame around code
        basicstyle=\small\ttfamily, % Use small true type font
        keywordstyle=[1]\color{Blue}\bf, % Perl functions bold and blue
        keywordstyle=[2]\color{Purple}, % Perl function arguments purple
        keywordstyle=[3]\color{Blue}\underbar, % Custom functions underlined and blue
        identifierstyle=, % Nothing special about identifiers                                         
        commentstyle=\usefont{T1}{pcr}{m}{sl}\color{MyDarkGreen}\small, % Comments small dark green courier font
        stringstyle=\color{Purple}, % Strings are purple
        showstringspaces=false, % Don't put marks in string spaces
        tabsize=5, % 5 spaces per tab
        %
        % Put standard Perl functions not included in the default language here
        morekeywords={rand},
        %
        % Put Perl function parameters here
        morekeywords=[2]{on, off, interp},
        %
        % Put user defined functions here
        morekeywords=[3]{test},
       	%
        morecomment=[l][\color{Blue}]{...}, % Line continuation (...) like blue comment
        numbers=left, % Line numbers on left
        firstnumber=1, % Line numbers start with line 1
        numberstyle=\tiny\color{Blue}, % Line numbers are blue and small
        stepnumber=5 % Line numbers go in steps of 5
}

\newcommand{\horrule}[1]{\rule{\linewidth}{#1}}

\newcommand\doubleplus{\ensuremath{\mathbin{+\mkern-10mu+}}}

% Creates a new command to include a perl script, the first parameter is the filename of the script (without .pl), the second parameter is the caption
\newcommand{\perlscript}[2]{
\begin{itemize}
\item[]\lstinputlisting[caption=#2,label=#1]{#1.cs}
\end{itemize}
}

\begin{document}

\begin{tabular}{l l}
\multirow{5}{*}{\includegraphics[width=2cm]{../../recursos/logo.png}} & Universidad del Istmo de Guatemala \\
 & Facultad de Ingenieria \\
 & Ing. en Sistemas \\
 & Informatica II \\
 & Prof. Ernesto Rodriguez - \href{mailto:erodriguez@unis.edu.gt}{erodriguez@unis.edu.gt} \\
\end{tabular}
\\\\\\

\begin{center}
        \horrule{0.5pt}
        \huge{Laboratorio \#5} \\
        \large{Fecha de entrega: 28 de Febrero, 2019 - 11:59pm} \\
        \horrule{1pt}
\end{center}

\emph{Instrucciones: Resolver cada uno de los ejercicios siguiendo sus respectivas
instrucciones. El trabajo debe ser entregado a traves de Github, en su repositorio del curso, colocado en una carpeta llamada "Laboratorio \#5".
Al menos que la pregunta indique diferente, todas las respuestas a preguntas escritas deben presentarse en
un documento formato pdf, el cual haya sido generado mediante Latex. Este laboratorio
debe ser elaborado en parejas.}

\section*{Tarea \#1 (25\%)}
\label{tarea1}

En C++ es costumbre separar las declaraciones de clases y las implementaciones de las mismas. Adjunto a este
laboratorio esta el archivo ``Vector2d.hh'', en el cual se declaran los \emph{metodos}
que debe implementar la clase ``Vector2d''. Utilize el archivo ``Vector2d.cc'' adjunto para
implementar los metodos de la clase ``Vector2d''. El constructor ha sido implementado
como ejemplo.
\\\\
El metodo to\_string de la clase vector debe retornar una \emph{cadena}
que sea una representacion del vector. Por ejemplo: ``$\langle 2,4.3 \rangle$''.

\section*{Tarea \#2 (25\%)}
Siguiendo el mismo patron de la \href{sec:tarea1}{Tarea \#1}, crear una clase llamada
``Vehiculo'' declarandola en un archivo de encabezados llamado ``Vehiculo.hh'' e
implementandola en un archivo de codigo llamado ``Vehiculo.cc''.

El constructor de esta clase no debe recibir ningun parametro, sin embargo
debe tener dos miembros \emph{privados} llamados ``Velocidad'' y ``Posici\'on''
de tipo ``Vector2d''. Ambos deben inicializarse con las coordenadas (0,0).
\\\\
Declarar dos metodos:
\begin{itemize}
        \item{$getVelocidad\ :\ \mathtt{void}\rightarrow\mathtt{Vector2d}$}
        \item{$getPosicion\ :\ \mathtt{void}\rightarrow\mathtt{Vector2d}$}
\end{itemize}
Estos metodos deben retornar la posici\'on y velocidad respectivamente. Por ulitmo,
declarar un metodo ``$to\_string\ :\ \mathtt{void}\rightarrow \mathtt{const\ std::string}$'' que retorne una representaci\'on legible del
vehiculo que muestre la velocidad y posici\'on de dicho vehiculo. Recuerde que debe utilizar
la directiva ``$\mathtt{\#include\ <string>}$'' para poder retornar valores de tipo \emph{string}.
Se recomienda que este metodo utilize el metodo ``to\_string'' de los vectores del vehiculo
para facilitar crear la cadena resultante.

\section*{Tarea \#3 (20\%)}
Declarar en la clase ``Vehiculo'' el metodo $acelerar\ :\ \mathtt{const\ Vector2d\&}\rightarrow\mathtt{const\ float}\rightarrow \mathtt{void}$.
Este metodo debe acelerar el vehiculo con la aceleraci\'on correspondiente al primer
parametro siendo ese vector el cambio de velocidad por segundo. Dicha aceleraci\'on
debe ser aplicada durante el tiempo indicado por el segundo parametro. Al finalizar la
ejecuci\'on de este metodo, tanto la velocidad como la posici\'on del vehiculo deben
actualizarse tal que correspondan a la aceleraci\'on que tuvo el vehiculo.
\\\\
Recuerde que la el archivo que declara la clase ``Vehiculo'' debe incluir la
directiva $\mathtt{\#include\ ``Vector2d.hh''}$ para poder utilizar la clase
''Vector2d''.

\section*{Tarea \#4 (20\%)}
Declarar en la clase ``Vehiculo'' el metodo ``$avanzar\ :\ \mathtt{const float}\rightarrow\mathtt{void}$''.
Este metodo debe utilizar la velocidad del vehiculo para moverlo durante la cantidad de tiempo que
se dio como parametro. Luego de la ejecuci\'on de este metodo, la posici\'on del vehiculo
debe corresponder al movimiento durante el tiempo dado como primer paramtero.

\section*{Tarea \#5 (10\%)}
Crear un archivo llamado ``main.cc''. En este arhivo debe declarar el
metodo \emph{main} a partir del cual se inica la ejecuci\'on del programa.
Este metodo debe hacer lo siguiente:
\begin{enumerate}
        \item{Crear una \emph{instancia} de un vehiculo}
        \item{Acelerarlo en la direcci\'on (3,1) durante 5 segundos}
        \item{Mover el vehiculo a velocidad constante durante 10 segundos}
        \item{Imprimir el estado actual del vehiculo utilizando su metodo ``to\_string''}
        \item{Acelerar el vehiculo en la direcci\'on (-7.2, 8) durante 4 segundos}
        \item{Mover el vehiculo a velocidad constante durante 9 segundos}
        \item{Imprimir el estado actual del vehiculo utilizando el metodo ``to\_string''} 
\end{enumerate}
El archivo ``main.cc'' debe incluir los archivos ``Vector2d.hh'' y ``Vehiculo.hh''
para poder utilizar las clases ``Vehiculo''y ``Vector2d''. El compilador (clang++)
necesita que todos los archivos de codigo sean especificados para llevar a cabo
la compilaci\'on por lo cual se debe compilar mediante el comando ``clang++ Vehiculo.cc 
Vector.cc main.cc''.

\end{document}