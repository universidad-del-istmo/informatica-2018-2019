\documentclass{beamer}
\usepackage{amsmath}
\usepackage[utf8]{inputenc}
\usepackage{hyperref}
\usepackage{multicol}
\usepackage{hyperref}
\usepackage{xcolor}
\usepackage{multirow}

\inputencoding{utf8}

\mode<presentation> {
    \usetheme{Madrid}
}

\usepackage{graphicx}
\usepackage{booktabs}

\title[Interpretadores]{Interpretadores Abstractos}
\author{Ernesto Rodriguez}
\institute{
    Universidad del Itsmo \\
    \medskip \textit{erodriguez@unis.edu.gt}
}

\date[\today]{}

\begin{document}

\begin{frame}
    \maketitle
\end{frame}

\begin{frame}
    \frametitle{¿Como se computa?}
    \begin{itemize}
    \item{Podemos definir funci\'ones mediante \emph{Tipos Abstractos}}
    \item{Sin embargo, necesitamos computar dichas funci\'ones.}
    \item{A todo esto ¿Que es computar?}
    \end{itemize}
\end{frame}

\begin{frame}
    \frametitle{¿Como se computa?}
    \begin{itemize}
    \item{Podemos definir funci\'ones mediante \emph{Tipos Abstractos}}
    \item{Sin embargo, necesitamos computar dichas funci\'ones.}
    \item{A todo esto ¿Que es computar?
    \begin{itemize}
        \item{Aplicar un conjunto de reglas para reducir un ADT.}
        \item{La computaci\'on termina cuando ya no se puede aplicar ninguna regla}
    \end{itemize}
    }
    \end{itemize}
\end{frame}

\begin{frame}
    \frametitle{¿Como se computa?}
    \begin{itemize}
        \item{Datao $\mathcal{A}:=\langle \mathcal{S}^0,\mathcal{D} \rangle$, llamamos
        al cuarteto $\langle f::\mathbb{A}\rightarrow\mathbb{R};\mathcal{R} \rangle$
        un {\bf procedimiento abstracto} ssi $\mathcal{R}$ es un conjunto de reglas
        para $f$.
        \begin{itemize}
            \item{$\mathbb{A}$ es el conjunto de parametros}
            \item{$\mathbb{R}$ es el conjunto de resultados}
        \end{itemize}
        }
        \item{La {\bf computaci\'on} de un procedimiento abstracto $p$ es
        la secuencia de {\bf constructores} $c_1\rightsquigarrow c_2\rightsquigarrow \ldots$
        segun las reglas de $p$}
        \item{Una {\bf computaci\'on abstracta} es una {\bf computacion} que ejecutamos
        en nuestra mente sin limites de memoria o tiempo.}
        \item{Un {\bf interpretador abstracto} es la maquina imaginaria que ejecuta
        dicha computaci\'on}
    \end{itemize}
\end{frame}

\begin{frame}
    \frametitle{Ejemplo: Las funciones $\rho$ y $\mathtt{concat}$}
        \tiny{
        $$
        \left\langle
            \begin{array}{ll}
            \rho::\mathcal{L}(\mathbb{N}) \rightarrow \mathcal{L}(\mathbb{N}); &
            {
                \left\{
                \begin{array}{l}
                    \rho(\mathtt{cons}(n,l)) \rightsquigarrow \mathtt{concat}(\rho(l), \mathtt{cons}(n,\mathtt{nil})) \\
                    \rho(\mathtt{nil}\rightsquigarrow \mathtt{nil})
                \end{array}
                \right\}
            }
            \end{array}
        \right\rangle
        $$
        \\
        $$
        \left\langle
            \begin{array}{ll}
            \mathtt{concat}::\mathcal{L}(\mathbb{N})\times\mathcal{L}(\mathbb{N}) \rightarrow \mathcal{L}(\mathbb{N}); &
            {
                \left\{
                \begin{array}{l}
                    \mathtt{concat}(\mathtt{cons}(n,l), r) \rightsquigarrow \mathtt{cons}(n, \mathtt{concat}(l,r)) \\
                    \mathtt{concat}(\mathtt{nil}\rightsquigarrow \mathtt{nil})
                \end{array}
                \right\}
            }
            \end{array}
        \right\rangle
        $$
        }
\end{frame}

\begin{frame}
\frametitle{Caracteristicas de los Procedimientos Abstractos}
\begin{itemize}
    \item{Se dice que un proceso abstracto {\bf termina} si se produce
    un termino que ninguna regla en $\mathcal{R}$ puede reducir}
    \item{Este modelo de computaci\'on esta inspirado en el
    calculo-$\lambda$ de Church.}
    \item{Los ``pasos'' }
\end{itemize}
\end{frame}

\end{document}