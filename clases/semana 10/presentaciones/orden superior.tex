\documentclass{beamer}
\usepackage{amsmath}
\usepackage[utf8]{inputenc}
\usepackage{hyperref}
\usepackage{multicol}
\usepackage{hyperref}
\usepackage{xcolor}
\usepackage{multicol}

\inputencoding{utf8}

\mode<presentation> {
    \usetheme{Madrid}
}

\usepackage{graphicx}
\usepackage{booktabs}

\title[Superior]{Funciones de Orden Superior}
\author{Ernesto Rodriguez}
\institute{
    Universidad del Itsmo \\
    \medskip \textit{erodriguez@unis.edu.gt}
}

\date[\today]{}

\begin{document}

\begin{frame}
    \maketitle
\end{frame}

\begin{frame}
\frametitle{Funciones de Orden Superior}
\begin{itemize}
    \item{Los lenguajes funcionales no diferencian
    funciones de valores}
    \item{Esto permite escribir funciones que aceptan
    otras funciones como parametro}
    \item{Este patron permite reemplazar muchos
    componentes tradicionales (ciclos, manejo de excepciones,
    saltos, etc.) mediante funciones.
    \begin{itemize}
        \item{Los Lambda Papers\cite{LambdaTheUltimate} describen
        diferentes formas de alcanzar esto.}
    \end{itemize}
    }
    \item{La mayoria de lenguajes modernos adopta este patron.
    \begin{itemize}
        \item{{\bf Java 8} por fin llego a donde Church estaba en 1930. -- \href{http://homepages.inf.ed.ac.uk/wadler/}{Phillip Wadler}}
    \end{itemize}
    }
    \item{Un patron llamado {\bf Programaci\'on Funcional Reactiva} permite crear
    interfaces graficas mediante {\bf funciones de orden superior}. Este Patron
    se utiliza en Elm, React y muchas otras herramientas.}
\end{itemize}
\end{frame}

\begin{frame}
\frametitle{Ejemplo: Composici\'on}
\begin{itemize}
    \item{¿Que {\bf tipo} deberia tener la composici\'on?}
    \item{¿Se puede programar la composici\'on en Elm?}
\end{itemize}
\end{frame}

\begin{frame}
\frametitle{Ciclos}
\begin{itemize}
    \item{Consideremos las siguientes funciones sobre listas:
    \begin{itemize}
        \item{$\mathtt{duplicar}\ :\ \mathcal{L}(\mathbb{Z})\rightarrow \mathcal{L}(\mathbb{Z})$}
        \item{$\mathtt{negar}\ :\ \mathcal{L}(\mathbb{B})\rightarrow \mathcal{L}(\mathbb{B})$}
    \end{itemize}
    }
    \item{¿Tiene algun problema la implementaci\'on?}
    \item{¿Podemos simplificar?}
\end{itemize}
\end{frame}

\begin{frame}
    \frametitle{Ciclos}
    \begin{itemize}
        \item{Consideremos las siguientes funciones sobre listas:
        \begin{itemize}
            \item{$\mathtt{duplicar}\ :\ \mathcal{L}(\mathbb{Z})\rightarrow \mathcal{L}(\mathbb{Z})$}
            \item{$\mathtt{negar}\ :\ \mathcal{L}(\mathbb{B})\rightarrow \mathcal{L}(\mathbb{B})$}
        \end{itemize}
        }
        \item{¿Tiene algun problema la implementaci\'on?
            \begin{itemize}
                \item{Repetici\'on de codigo}
            \end{itemize}
        }
        \item{¿Podemos simplificar?
            \begin{itemize}
                \item{Idea: Abstraer la recursi\'on}
            \end{itemize}
        }
    \end{itemize}
\end{frame}

\begin{frame}
    \frametitle{Ciclos}
    La funci\'on $\mathtt{map}\ :\ (a\rightarrow b) \rightarrow
    \mathcal{L}(a) \rightarrow \mathcal{L}(b)$ hace exactamente eso:
\end{frame}

\begin{frame}
    \frametitle{Ciclos}
    La funci\'on $\mathtt{map}\ :\ (a\rightarrow b) \rightarrow
    \mathcal{L}(a) \rightarrow \mathcal{L}(b)$ hace exactamente eso:
    $$
    \mathtt{map}\ lista\ f=\left\{
        \begin{array}{l l}
            [\ ] & \mbox{if }lista\equiv [\ ] \\
            \mathtt{f\ x::\mathtt{map}\ xs\ f} & \mbox{if }lista\equiv x::xs
        \end{array}
    \right.
    $$
    ¿Como implementamos $\mathtt{duplicar}$ y $\mathtt{negar}$ utilizando $\mathtt{map}$?
\end{frame}

\begin{frame}
    \frametitle{Ciclos}
    La funci\'on $\mathtt{map}\ :\ (a\rightarrow b) \rightarrow
    \mathcal{L}(a) \rightarrow \mathcal{L}(b)$ hace exactamente eso:
    $$
    \mathtt{map}\ f\ lista=\left\{
        \begin{array}{l l}
            [\ ] & \mbox{if }lista\equiv [\ ] \\
            \mathtt{f\ x::\mathtt{map}\ f\ xs} & \mbox{if }lista\equiv x::xs
        \end{array}
    \right.
    $$
    ¿Como implementamos $\mathtt{duplicar}$ y $\mathtt{negar}$ utilizando $\mathtt{map}$?
    \begin{itemize}
        \item{$\mathtt{duplicar}=\mathtt{map}\ ((*)\ 2)$}
        \item{$\mathtt{negar}=\mathtt{map}\ \mathtt{not}$}
    \end{itemize}
\end{frame}

\begin{frame}
\frametitle{Funciones Anonimas y Let}
\begin{itemize}
    \item{A menudo se definen funci\'ones de uso
    unico.}
    \item{Es conveniente definir la funci\'on en
    donde sera utilizada}
    \item{A veces se necesita contexto local para
    definir la funci\'on.}
    \item{Ejemplo: $\mathtt{multiplicar}\ :\ \mathbb{Z}\rightarrow \mathcal{L}(\mathtt{Z})\rightarrow \mathcal{L}(\mathtt{Z})$}
\end{itemize}
\end{frame}

\begin{frame}
    \frametitle{Funciones Anonimas y Let}
    \begin{itemize}
        \item{A menudo se definen funci\'ones de uso
        unico.}
        \item{Es conveniente definir la funci\'on en
        donde sera utilizada}
        \item{A veces se necesita contexto local para
        definir la funci\'on.}
        \item{Ejemplo: $\mathtt{multiplicar}\ :\ \mathbb{Z}\rightarrow \mathcal{L}(\mathtt{Z})
        \rightarrow \mathcal{L}(\mathtt{Z})$}
    \end{itemize}
    \begin{tabular}{l l}
        $
        \begin{array}{l l l}
            \multicolumn{3}{l}{\mathtt{multiplicar}\ factor\ lista=} \\
            & \multicolumn{2}{l}{\mathtt{let}} \\
            & & \mathtt{op}\ valor=factor*valor \\
            & \multicolumn{2}{l}{\mathtt{in}} \\
            & & \mathtt{map}\ op\ lista
        \end{array}
        $
        &
        $
        \begin{array}{l l}
            \multicolumn{2}{l}{\mathtt{multiplicar}\ factor\ lista=} \\
            & \mathtt{map}\ (\lambda valor\rightarrow factor*valor)\ lista
        \end{array}
        $
        \\
    \end{tabular}
\end{frame}

\begin{frame}
    \frametitle{Generalizando los ciclos de Listas}
    \begin{itemize}
        \item{¿Existen funciones que no se pueden expresar mediante $\mathtt{map}?$}
    \end{itemize}
\end{frame}

\begin{frame}
    \frametitle{Generalizando los ciclos de Listas}
    \begin{itemize}
        \item{¿Existen funciones que no se pueden expresar mediante $\mathtt{map}?$
        \begin{itemize}
            \item{¿$\mathtt{count}\ :\ \mathcal{L}(a)\rightarrow \mathbb{N}$?}
            \item{¿$\mathtt{sumatoria}\ :\ \mathcal{L}(\mathbb{Z})\rightarrow \mathbb{Z}$?}
        \end{itemize}
        }
    \end{itemize}
\end{frame}

\begin{frame}
    \frametitle{Generalizando los ciclos de Listas}
    \begin{itemize}
        \item{¿Existen funciones que no se pueden expresar mediante $\mathtt{map}?$
        \begin{itemize}
            \item{¿$\mathtt{count}\ :\ \mathcal{L}(a)\rightarrow \mathbb{N}$?}
            \item{¿$\mathtt{sumatoria}\ :\ \mathcal{L}(\mathbb{Z})\rightarrow \mathbb{Z}$?}
        \end{itemize}
        }
        \item{El tipo de la funci\'on esta incorrecto, ya que $\mathtt{map}$ solo
        puede producir listas.}
    \end{itemize}
\end{frame}

\begin{frame}
    \frametitle{Primer intento}
    Llamaremos a la nueva funci\'on $\mathtt{fold}\ :\ (a\rightarrow b)\rightarrow
        \mathcal{L}(a)\rightarrow b$:
    $$
    \mathtt{fold}\ f\ lista = \left\{
        \begin{array}{ll}
            ? & \mbox{if }lista\equiv [\ ] \\
            f\ x & \mbox{if }lista\equiv x::xs
        \end{array}
    \right.
    $$
    \begin{itemize}
        \item{{\bf Problema: }No es posible retornar un valor cuando $lista$ esta vacia.}
    \end{itemize}
\end{frame}

\begin{frame}
    \frametitle{Primer intento}
    Llamaremos a la nueva funci\'on $\mathtt{fold}\ :\ (a\rightarrow b)\rightarrow
        {\color{red} b} \rightarrow \mathcal{L}(a)\rightarrow b$:
    $$
    \mathtt{fold}\ f\ {\color{red} base}\ lista=\left\{
        \begin{array}{ll}
            {\color{red} base} & \mbox{if }lista\equiv [\ ] \\
            f\ x & \mbox{if }lista\equiv x::xs
        \end{array}
    \right.
    $$
    \begin{itemize}
        \item{{\bf Problema: }No es posible retornar un valor cuando $lista$ esta vacia.
        \begin{itemize}
            \item{Aceptamos un valor de retorno para cubrir ese caso}
        \end{itemize}
        }
    \end{itemize}
\end{frame}

\begin{frame}
    \frametitle{Segundo intento}
    Llamaremos a la nueva funci\'on $\mathtt{fold}\ :\ (a\rightarrow b)\rightarrow
        b \rightarrow \mathcal{L}(a)\rightarrow b$:
    $$
    \mathtt{fold}\ f\ base\ lista=\left\{
        \begin{array}{ll}
            base & \mbox{if }lista\equiv [\ ] \\
            f\ x & \mbox{if }lista\equiv x::xs
        \end{array}
    \right.
    $$
    \begin{itemize}
        \item{{\bf Problema: }No es posible retornar un valor cuando $lista$ esta vacia.
        \begin{itemize}
            \item{Aceptamos un valor de retorno para cubrir ese caso}
        \end{itemize}
        }
        \item{{\bf Problema: }El valor de retorno solo depende del primer valor.}
    \end{itemize}
\end{frame}

\begin{frame}
\frametitle{Segundo intento}
{\bf Idea:}
\begin{itemize}
    \item{``Recordar'' el valor producido por cada elemento de la lista}
    \item{Permitir que la {\bf funcion de orden superior} o el {\bf reductor}
    utilize ese valor para producir un nuevo valor}
    \item{Retornar el ultimo valor que haya sido calculado.}
\end{itemize}
\end{frame}

\begin{frame}
    \frametitle{Segundo intento}
    Llamaremos a la nueva funci\'on $\mathtt{fold}\ :\ ({\color{red} b}\rightarrow a\rightarrow b)\rightarrow
        b \rightarrow \mathcal{L}(a)\rightarrow b$:
    $$
    \mathtt{fold}\ f\ base\ lista=\left\{
        \begin{array}{ll}
            base & \mbox{if }lista\equiv [\ ] \\
            {\color{red} f\ (\mathtt{fold}\ f\ base\ xs)\ x} & \mbox{if }lista\equiv x::xs
        \end{array}
    \right.
    $$
    \begin{itemize}
        \item{{\bf Problema: }No es posible retornar un valor cuando $lista$ esta vacia.
        \begin{itemize}
            \item{Aceptamos un valor de retorno para cubrir ese caso}
        \end{itemize}
        }
        \item{{\bf Problema: }El valor de retorno solo depende del primer valor.
            \begin{itemize}
                \item{Recordar el valor que fue calculado en el paso anterior}
            \end{itemize}
        }
    \end{itemize}
\end{frame}

\begin{frame}
    \frametitle{Fold: Recursi\'on Generalizada}
    \begin{itemize}
        \item{¿Podemos implementar la funci\'on $\mathtt{count}?$}
        \item{¿Podemos implementar la funci\'on $\mathtt{sumatoria}$?}
        \item{¿Podemos implementar la funci\'on $\mathtt{map}$?}
    \end{itemize}
    
\end{frame}

\begin{frame}
    \frametitle{Fold: Recursi\'on Generalizada}
    \begin{itemize}
        \item{¿Podemos implementar la funci\'on $\mathtt{count}?$
            \begin{itemize}
                \item{$\mathtt{sumatoria}\ xs=\mathtt{fold}\ (\lambda\ a\ \_\rightarrow a+1)
                    \ 0\ xs$}
            \end{itemize}
        }
        \item{¿Podemos implementar la funci\'on $\mathtt{sumatoria}$?}
        \item{¿Podemos implementar la funci\'on $\mathtt{map}$?}
    \end{itemize}
    
\end{frame}

\begin{frame}
    \frametitle{Fold: Recursi\'on Generalizada}
    \begin{itemize}
        \item{¿Podemos implementar la funci\'on $\mathtt{count}?$
            \begin{itemize}
                \item{$\mathtt{sumatoria}\ xs=\mathtt{fold}\ (\lambda\ a\ \_\rightarrow a+1)
                    \ 0\ xs$}
            \end{itemize}
        }
        \item{¿Podemos implementar la funci\'on $\mathtt{sumatoria}$?
            \begin{itemize}
                \item{$\mathtt{sumatoria}\ xs=\mathtt{fold}\ (\lambda\ a\ b\rightarrow a+b)
                    \ 0\ xs$}
            \end{itemize}
        }
        \item{¿Podemos implementar la funci\'on $\mathtt{map}$?}
    \end{itemize}
    
\end{frame}

\begin{frame}
    \frametitle{Fold: Recursi\'on Generalizada}
    \begin{itemize}
        \item{¿Podemos implementar la funci\'on $\mathtt{count}?$
            \begin{itemize}
                \item{$\mathtt{sumatoria}\ xs=\mathtt{fold}\ (\lambda\ a\ \_\rightarrow a+1)
                    \ 0\ xs$}
            \end{itemize}
        }
        \item{¿Podemos implementar la funci\'on $\mathtt{sumatoria}$?
            \begin{itemize}
                \item{$\mathtt{sumatoria}\ xs=\mathtt{fold}\ (\lambda\ a\ b\rightarrow a+b)
                    \ 0\ xs$}
            \end{itemize}
        }
        \item{¿Podemos implementar la funci\'on $\mathtt{map}$?
            \begin{itemize}
                \item{$\mathtt{map}\ f\ as=\mathtt{fold}\ (\lambda\ bs\ a\rightarrow f\ a::bs)\ [\ ]\ as$}
            \end{itemize}
        }
        \item{La funci\'on $\mathtt{fold}$ generaliza todas las {\bf funciones recursivas}
        sobre listas. Ver ``Functional Programming with Bananas, Lenses, Envelopes and Barbed Wire''
        \cite{Meijer91functionalprogramming}.}
        \item{La generalizaci\'on de la funci\'on $\mathtt{fold}$ para cualquier tipo
        se conoce como un {\bf catamorfismo}.}
    \end{itemize}
\end{frame}

\begin{frame}
    \frametitle{Referencias}
    \bibliography{../../../recursos/referencias}
    \bibliographystyle{plain}
\end{frame}

\end{document}