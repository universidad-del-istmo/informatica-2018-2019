%%%%%%%%%%%%%%%%%%%%%%%%%%%%%%%%%%%%%%%%%
% Programming/Coding Assignment
% LaTeX Template
%
% This template has been downloaded from:
% http://www.latextemplates.com
%
% Original author:
% Ted Pavlic (http://www.tedpavlic.com)
%
% Note:
% The \lipsum[#] commands throughout this template generate dummy text
% to fill the template out. These commands should all be removed when 
% writing assignment content.
%
% This template uses a Perl script as an example snippet of code, most other
% languages are also usable. Configure them in the "CODE INCLUSION 
% CONFIGURATION" section.
%
%%%%%%%%%%%%%%%%%%%%%%%%%%%%%%%%%%%%%%%%%

%----------------------------------------------------------------------------------------
%	PACKAGES AND OTHER DOCUMENT CONFIGURATIONS
%----------------------------------------------------------------------------------------

\documentclass{article}

\usepackage{fancyhdr} % Required for custom headers
\usepackage{lastpage} % Required to determine the last page for the footer
\usepackage{extramarks} % Required for headers and footers
\usepackage[usenames,dvipsnames]{color} % Required for custom colors
\usepackage{graphicx} % Required to insert images
\usepackage{listings} % Required for insertion of code
\usepackage{courier} % Required for the courier font
\usepackage{multirow}
\usepackage{hyperref}
\usepackage{amsmath}
\usepackage{amssymb}

% Margins
\topmargin=-0.45in
\evensidemargin=0in
\oddsidemargin=0in
\textwidth=6.5in
\textheight=9.0in
\headsep=0.25in

\linespread{1.1} % Line spacing

%----------------------------------------------------------------------------------------
%	CODE INCLUSION CONFIGURATION
%----------------------------------------------------------------------------------------

\definecolor{MyDarkGreen}{rgb}{0.0,0.4,0.0} % This is the color used for comments
\lstloadlanguages{c} % Load Perl syntax for listings, for a list of other languages supported see: ftp://ftp.tex.ac.uk/tex-archive/macros/latex/contrib/listings/listings.pdf
\lstset{language=[sharp]c, % Use Perl in this example
        frame=single, % Single frame around code
        basicstyle=\small\ttfamily, % Use small true type font
        keywordstyle=[1]\color{Blue}\bf, % Perl functions bold and blue
        keywordstyle=[2]\color{Purple}, % Perl function arguments purple
        keywordstyle=[3]\color{Blue}\underbar, % Custom functions underlined and blue
        identifierstyle=, % Nothing special about identifiers                                         
        commentstyle=\usefont{T1}{pcr}{m}{sl}\color{MyDarkGreen}\small, % Comments small dark green courier font
        stringstyle=\color{Purple}, % Strings are purple
        showstringspaces=false, % Don't put marks in string spaces
        tabsize=5, % 5 spaces per tab
        %
        % Put standard Perl functions not included in the default language here
        morekeywords={rand},
        %
        % Put Perl function parameters here
        morekeywords=[2]{on, off, interp},
        %
        % Put user defined functions here
        morekeywords=[3]{test},
       	%
        morecomment=[l][\color{Blue}]{...}, % Line continuation (...) like blue comment
        numbers=left, % Line numbers on left
        firstnumber=1, % Line numbers start with line 1
        numberstyle=\tiny\color{Blue}, % Line numbers are blue and small
        stepnumber=5 % Line numbers go in steps of 5
}

\newcommand{\horrule}[1]{\rule{\linewidth}{#1}}

% Creates a new command to include a perl script, the first parameter is the filename of the script (without .pl), the second parameter is the caption
\newcommand{\perlscript}[2]{
\begin{itemize}
\item[]\lstinputlisting[caption=#2,label=#1]{#1.cs}
\end{itemize}
}

\begin{document}

\begin{tabular}{l l}
\multirow{5}{*}{\includegraphics[width=2cm]{../../recursos/logo.png}}
 & Universidad del Istmo de Guatemala \\
 & Facultad de Ingenieria \\
 & Ing. en Sistemas \\
 & Informatica 1 \\
 & Prof. Ernesto Rodriguez - \href{mailto:erodriguez@unis.edu.gt}{erodriguez@unis.edu.gt} \\
\end{tabular}
\\\\\\

\begin{center}
        \horrule{0.5pt}
        \huge{Hoja de trabajo \#4} \\
        \large{Fecha de entrega: 30 de Agosto, 2018 - 11:59pm} \\
        \horrule{1pt}
\end{center}

\emph{Instrucciones: Resolver cada uno de los ejercicios siguiendo sus respectivas
instrucciones. El trabajo debe ser entregado a traves de Github, en su repositorio del curso, colocado en una carpeta llamada "Hoja de trabajo 1".
Al menos que la pregunta indique diferente, todas las respuestas a preguntas escritas deben presentarse en
un documento formato pdf, el cual haya sido generado mediante Latex. }

\section*{Ejercicio \#1 (10\%)}
A continuaci\'on se le presentara una serie de definiciones de conjuntos pertenecientes al
conjunto $2^{\mathbb{N}}$. Indicar que definiciones corresponden al mismo conjunto, es decir
que definiciones definen conjuntos que tienen los mismos elementos.
\begin{enumerate}
        \item{$a:=\{1,2,4,8,16,32,64\}$}
        \item{$b:=\{n\ \in \mathbb{N}\ |\ \exists x \in \mathbb{N}\ .\ x=n/5 \}$}
        \item{$c:=\{n\in \mathbb{N}\ |\ \exists x\in\mathbb{N}\ .\ n=x*x \}$}
        \item{$d:=\{n\in\mathbb{N}\ |\ \exists i\in\mathbb{N}\ .\ n=2^i\wedge n<100 \}$}
        \item{$e:=\{ n\in\mathbb{N}\ |\ \exists x\in \mathbb{N}\ .\ x=\sqrt{n} \}$}
        \item{$f:=\{ n\in\mathbb{N}\ |\ \exists x\in \mathbb{N}\ .\ n=x+x+x+x+x \}$}
\end{enumerate}

\section*{Ejercicio \#2 (10\%)}
Utilize la \emph{jerga matematica} para definir los siguientes conjuntos:
\begin{enumerate}
        \item{El conjunto de todos los naturales divisibles dentro de $5$}
        \item{El conjunto de todos los naturales divisibles dentro de $4$ y $5$}
        \item{El conjunto de todos los naturales que son primos}
        \item{El conjunto de todos los conjuntos de numeros naturales que contienen
        un numero divisible dentro de $15$}
        \item{El conjunto de todos los conjuntos de numeros naturales que al ser sumados
        producen $42$ como resultado}
\end{enumerate}

\section*{Ejercicio \#3 (10\%)}

Un numero \emph{semi-primo} es el producto de dos numeros primos. Los numeros
\emph{semiprimos} tienen la peculiaridad que nada m\'as son divisibles
entre $1$ y los dos primos de los cuales dicho numero es un producto. Un ejemplo
es el numero seis ($6=2*3$) el cual se obtiene al multiplicar los primos $2$ y $3$.
\\\\
Definir una relaci\'on llamada $S\subset \mathbb{N}_{50}\times\mathbb{N}_{50}\times\mathbb{N}_{50}$ en
donde $\mathbb{N}_{30}:=\{ n \in \mathbb{N}\ |\ n\leq 30 \}$. La cual relaciona a todos los numeros
\emph{semi-primos} menores a $30$ con los numeros primos que lo forman. Las tripletas que pertencen
al conjunto que define dicha relaci\'on deben ser de la forma $\langle \mathtt{primo}_1,\mathtt{primo}_2,
\mathtt{semi-primo} \rangle$, por ejemplo, para el numero $6$ corresponderia la tripleta $\langle 2,3,6 \rangle$

\section*{Ejercicio \#4 (20\%)}
Utilize la \emph{jerga matematica} para definir los conjuntos a los que corresponden las
siguientes funci\'ones:
\begin{enumerate}
        \item{$f:\mathbb{N}\rightarrow\mathbb{N}$; $f(x)=x+x$}
        \item{$g:\mathbb{N}\rightarrow\mathbb{B}$; $g(x)$ es verdadero si
        $x$ es divisible dentro de $5$, falso en caso contrario. Nota: $\mathbb{B}=
        \{\mathtt{true},\mathtt{false}\}$, puede definir dos conjuntos separados y
        definir la funci\'on como la union de ambos conjuntos.}
        \item{Indicar el conjunto al que pertenece la funci\'on $f\circ g$}
        \item{Definir el conjunto que corresponde a la funci\'on $f\circ g$}
\end{enumerate}

\section*{Ejercicio \#5 (20\%)}
Dadas las siguientes funciones que pertenecen a $\mathbb{R}\rightarrow \mathbb{R}$, indique
si la funci\'on es injectiva, surjectiva o bijectiva.
\begin{enumerate}
        \item{$f(x)=x^2$}
        \item{$g(x)=\frac{1}{cos(x-1)}$}
        \item{$h(x)=2x$}
        \item{$w(x)=x+1$}
\end{enumerate}

\section*{Ejercicio \#6 (30\%)}
A continuaci\'on se definira una bijecci\'on entre los numeros naturales ($\mathbb{N}$) y los
numeros enteros ($\mathbb{Z}$). Se utilizaran varios conjuntos intermediariarios para facilitar
el proceso.
\begin{enumerate}
        \item{Definir el conjunto $B_1\in \mathbb{N}\times\mathbb{N}$ el cula empareja a los
        numeros naturales \emph{pares} con todos los naturales mayores a $0$. Eg. $B_1=\{
        \langle 2,1 \rangle, \langle 4,2 \rangle, \langle 6, 3 \rangle\ldots \}$}
        \item{Definir el conjunto $B_{2a}\in \mathbb{N}\times\mathbb{N}$ el cula empareja a los
        numeros naturales \emph{impares} con todos los naturales mayores a $0$. Eg. $B_{2a}=\{
        \langle 1,1 \rangle, \langle 3,2 \rangle, \langle 5, 3 \rangle\ldots \}$}
        \item{Definir el conjunto $B_{2}\in \mathbb{N}\times\mathbb{Z}$ el cual se definie
        exactamente igual al conjunto $B_{2a}$ excepto que los valores en el contradominio
        son negativos}
        \item{El conjutno $B:= \{\langle 0,0\rangle \}\cup B_{1} \cup B_{2}$ es la bijeccion
        que se intenta definir.}
\end{enumerate}

% \bibliography{../../recursos/referencias}
% \bibliographystyle{plain}

\end{document}