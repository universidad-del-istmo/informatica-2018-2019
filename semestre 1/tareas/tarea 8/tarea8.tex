%%%%%%%%%%%%%%%%%%%%%%%%%%%%%%%%%%%%%%%%%
% Programming/Coding Assignment
% LaTeX Template
%
% This template has been downloaded from:
% http://www.latextemplates.com
%
% Original author:
% Ted Pavlic (http://www.tedpavlic.com)
%
% Note:
% The \lipsum[#] commands throughout this template generate dummy text
% to fill the template out. These commands should all be removed when 
% writing assignment content.
%
% This template uses a Perl script as an example snippet of code, most other
% languages are also usable. Configure them in the "CODE INCLUSION 
% CONFIGURATION" section.
%
%%%%%%%%%%%%%%%%%%%%%%%%%%%%%%%%%%%%%%%%%

%----------------------------------------------------------------------------------------
%	PACKAGES AND OTHER DOCUMENT CONFIGURATIONS
%----------------------------------------------------------------------------------------

\documentclass{article}

\usepackage{fancyhdr} % Required for custom headers
\usepackage{lastpage} % Required to determine the last page for the footer
\usepackage{extramarks} % Required for headers and footers
\usepackage[usenames,dvipsnames]{color} % Required for custom colors
\usepackage{graphicx} % Required to insert images
\usepackage{listings} % Required for insertion of code
\usepackage{courier} % Required for the courier font
\usepackage{multirow}
\usepackage{hyperref}
\usepackage{amsmath}
\usepackage{amssymb}

% Margins
\topmargin=-0.45in
\evensidemargin=0in
\oddsidemargin=0in
\textwidth=6.5in
\textheight=9.0in
\headsep=0.25in

\linespread{1.1} % Line spacing

%----------------------------------------------------------------------------------------
%	CODE INCLUSION CONFIGURATION
%----------------------------------------------------------------------------------------

\definecolor{MyDarkGreen}{rgb}{0.0,0.4,0.0} % This is the color used for comments
\lstloadlanguages{c} % Load Perl syntax for listings, for a list of other languages supported see: ftp://ftp.tex.ac.uk/tex-archive/macros/latex/contrib/listings/listings.pdf
\lstset{language=[sharp]c, % Use Perl in this example
        frame=single, % Single frame around code
        basicstyle=\small\ttfamily, % Use small true type font
        keywordstyle=[1]\color{Blue}\bf, % Perl functions bold and blue
        keywordstyle=[2]\color{Purple}, % Perl function arguments purple
        keywordstyle=[3]\color{Blue}\underbar, % Custom functions underlined and blue
        identifierstyle=, % Nothing special about identifiers                                         
        commentstyle=\usefont{T1}{pcr}{m}{sl}\color{MyDarkGreen}\small, % Comments small dark green courier font
        stringstyle=\color{Purple}, % Strings are purple
        showstringspaces=false, % Don't put marks in string spaces
        tabsize=5, % 5 spaces per tab
        %
        % Put standard Perl functions not included in the default language here
        morekeywords={rand},
        %
        % Put Perl function parameters here
        morekeywords=[2]{on, off, interp},
        %
        % Put user defined functions here
        morekeywords=[3]{test},
       	%
        morecomment=[l][\color{Blue}]{...}, % Line continuation (...) like blue comment
        numbers=left, % Line numbers on left
        firstnumber=1, % Line numbers start with line 1
        numberstyle=\tiny\color{Blue}, % Line numbers are blue and small
        stepnumber=5 % Line numbers go in steps of 5
}

\newcommand{\horrule}[1]{\rule{\linewidth}{#1}}

% Creates a new command to include a perl script, the first parameter is the filename of the script (without .pl), the second parameter is the caption
\newcommand{\perlscript}[2]{
\begin{itemize}
\item[]\lstinputlisting[caption=#2,label=#1]{#1.cs}
\end{itemize}
}

\begin{document}

\begin{tabular}{l l}
\multirow{5}{*}{\includegraphics[width=2cm]{../../recursos/logo.png}}
 & Universidad del Istmo de Guatemala \\
 & Facultad de Ingenieria \\
 & Ing. en Sistemas \\
 & Informatica 1 \\
 & Prof. Ernesto Rodriguez - \href{mailto:erodriguez@unis.edu.gt}{erodriguez@unis.edu.gt} \\
\end{tabular}
\\\\\\

\begin{center}
        \horrule{0.5pt}
        \huge{Hoja de trabajo \#8} \\
        \large{Fecha de entrega: 11 de Octubre, 2018 - 11:59pm} \\
        \horrule{1pt}
\end{center}

\emph{Instrucciones: Resolver cada uno de los ejercicios siguiendo sus respectivas
instrucciones. El trabajo debe ser entregado a traves de Github, en su repositorio del curso, colocado en una
carpeta llamada "Hoja de trabajo 8". Al menos que la pregunta indique diferente, todas las
respuestas a preguntas escritas deben presentarse en un documento formato pdf, el cual
haya sido generado mediante Latex. }\\\\

{\bf Nota: }En este deber se omitira la ubicaci\'on exacta del compilador
de elm, y solo se escribira elm. Por ejemplo, en vez de escribir:\\\\
$>\mathtt{node\_modules}\backslash\mathtt{elm\ repl}$\\\\
Se escribira:\\\\
$>\mathtt{elm\ repl}$\\\\
Adicionalmente, asegurarse que las funciones y modulos que sean declarados
en su deber correspondan exactamente a los nombres escritos en dicho deber
ya que se utilizaran pruebas automatizadas para calificar.
\\\\
\section*{Ejercicio \#1 (30\%)}
Defina la funci\'on $\mathtt{zipWith}\ :\ (a\rightarrow b\rightarrow c)\rightarrow \mathtt{List}\
a\rightarrow \mathtt{List}\ b\rightarrow \mathtt{List}\ c$ la cual debe aceptar una funci\'on
como parametro y utilizar dicha funci\'on para combinar los valores de las dos listas pasadas
como parametro y producir una lista con la combinaci\'on de los valores.

\section*{Ejercicio \#2 (35\%)}
Defina la funci\'on $\mathtt{groupBy}\ :\ (a\rightarrow \mathtt{Bool})\rightarrow \mathtt{List}\ a
\rightarrow(\mathtt{List}\ a,\mathtt{List}\ a)$ la cual acepta una funci\'on que actua de clasificador
colocando en la la primera lista del resultado todos los elementos para los cuales dicha funci\'on
es $\mathtt{True}$ y en la segunda los elementos para los cuales la funci\'on retorna $\mathtt{False}$.
Ejemplo:
\begin{itemize}
        \item{$\mathtt{groupBy}\ (\lambda x\rightarrow x < 3)\ [1,2,3,4]$
        \begin{itemize}
                \item{$([1,2],[3,4])$}
        \end{itemize}
        }
\end{itemize}

\section*{Ejercicio \#3 (35\%)}
Elm trae integrado un tipo llamado $\mathtt{Maybe}$ el cual tiene dos constructores: '$\mathtt{Just}
\ :\ a \rightarrow \mathtt{Maybe}\ a$' y '$\mathtt{Nothing}\ :\ \mathtt{Maybe}\ a$'. Este tipo se
utiliza para representar valores opcionales o condiciones erroneas.
\\\\
Su tarea es definir la funci\'on $\mathtt{bind}\ :\ \mathtt{Maybe}\ a\rightarrow (a\rightarrow \mathtt{Maybe}\ b)
\rightarrow \mathtt{Maybe}\ b$ la cual toma como primer parametro un valor de tipo $\mathtt{Maybe}\ a$, el cual
es un valor opcional, una funci\'on que transforma el valor $a$ a un $\mathtt{Maybe}\ b$ y produce un
$\mathtt{Maybe}\ b$. Si el primer parametro es $\mathtt{Nothing}$, la funci\'on simplemente debe retornar
nothing. Ejemplos de su utilizaci\'on:
\begin{itemize}
        \item{$\mathtt{bind}\ (\mathtt{Just}\ 5)\ (\lambda i\rightarrow \mathtt{Just}\ (i+3))$
                \begin{itemize}
                        \item{$\mathtt{Just}\ 8$}
                \end{itemize}
        }
        \item{$\mathtt{bind}\ \mathtt{Nothing}\ (\lambda i\rightarrow \mathtt{Just\ (i+3)})$
                \begin{itemize}
                        \item{$\mathtt{Noting}$}
                \end{itemize}
        }
\end{itemize}


\end{document}